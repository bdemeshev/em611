\documentclass[12pt]{article} % размер шрифта

\usepackage{tikz} % картинки в tikz
\usepackage{microtype} % свешивание пунктуации

\usepackage{array} % для столбцов фиксированной ширины

\usepackage{url} % для вставки ссылок \url{...}

\usepackage{indentfirst} % отступ в первом параграфе

\usepackage{sectsty} % для центрирования названий частей
\allsectionsfont{\centering} % приказываем центрировать все sections

\usepackage{amsthm} % теоремы и доказательства

\theoremstyle{definition} % прямой шрифт в условии теорем
\newtheorem{theorem}{Теорема}[section]


\usepackage{amsmath, amssymb} % куча стандартных математических плюшек

\usepackage[top=2cm, left=1.5cm, right=1.5cm, bottom=2cm]{geometry} % размер текста на странице

\usepackage{lastpage} % чтобы узнать номер последней страницы

\usepackage{enumitem} % дополнительные плюшки для списков
%  например \begin{enumerate}[resume] позволяет продолжить нумерацию в новом списке
\usepackage{caption} % подписи к картинкам без плавающего окружения figure


\usepackage{fancyhdr} % весёлые колонтитулы
\pagestyle{fancy}
\lhead{Эконометрика, финтех}
\chead{}
\rhead{2020-12-21, большой праздник}
\lfoot{}
\cfoot{}
\rfoot{\thepage/\pageref{LastPage}}
\renewcommand{\headrulewidth}{0.4pt}
\renewcommand{\footrulewidth}{0.4pt}



\usepackage{todonotes} % для вставки в документ заметок о том, что осталось сделать
% \todo{Здесь надо коэффициенты исправить}
% \missingfigure{Здесь будет картина Последний день Помпеи}
% команда \listoftodos — печатает все поставленные \todo'шки

\usepackage{booktabs} % красивые таблицы
% заповеди из документации:
% 1. Не используйте вертикальные линии
% 2. Не используйте двойные линии
% 3. Единицы измерения помещайте в шапку таблицы
% 4. Не сокращайте .1 вместо 0.1
% 5. Повторяющееся значение повторяйте, а не говорите "то же"

\usepackage{fontspec} % поддержка разных шрифтов
\usepackage{polyglossia} % поддержка разных языков

\setmainlanguage{russian}
\setotherlanguages{english}

\setmainfont{Linux Libertine O} % выбираем шрифт
% если Linux Libertine не установлен, то
% можно также попробовать Helvetica, Arial, Cambria и т.Д.

% чтобы использовать шрифт Linux Libertine на личном компе,
% его надо предварительно скачать по ссылке
% http://www.linuxlibertine.org/index.php?id=91&L=1

% на сервисах типа sharelatex.com этот шрифт есть :)

\newfontfamily{\cyrillicfonttt}{Linux Libertine O}
% пояснение зачем нужно шаманство с \newfontfamily
% http://tex.stackexchange.com/questions/91507/

\AddEnumerateCounter{\asbuk}{\russian@alph}{щ} % для списков с русскими буквами
\setlist[enumerate, 2]{label=\asbuk*),ref=\asbuk*} % списки уровня 2 будут буквами а) б) ...

%% эконометрические и вероятностные сокращения
\DeclareMathOperator{\Cov}{Cov}
\DeclareMathOperator{\sCov}{sCov}
\DeclareMathOperator{\sVar}{sVar}
\DeclareMathOperator{\sCorr}{sCorr}

\DeclareMathOperator{\Corr}{Corr}
\DeclareMathOperator{\Var}{Var}
\DeclareMathOperator{\E}{E}
\def \hb{\hat{\beta}}
\def \hs{\hat{\sigma}}
\def \htheta{\hat{\theta}}
\def \s{\sigma}
\def \hy{\hat{y}}
\def \hY{\hat{Y}}
\def \v1{\vec{1}}
\def \e{\varepsilon}
\def \he{\hat{\e}}
\def \z{z}
\def \hVar{\widehat{\Var}}
\def \hCorr{\widehat{\Corr}}
\def \hCov{\widehat{\Cov}}
\def \cN{\mathcal{N}}

\let\P\relax
\DeclareMathOperator{\P}{\mathbb{P}}


\begin{document}


Дорогой храбрый воин или храбрая воительница! Удачи тебе на большом празднике по эконометрике!
Начни с того, что напиши клятву и подпишись под ней:

\vspace{10pt}
\textit{Я клянусь честью студента, что буду выполнять эту работу самостоятельно.}
\vspace{10pt}


А теперь — задачки:


\begin{enumerate}
 \item Исходная выборка $y$ — вектор из $n$ независимых случайных величин, 
равномерных на $[0;1]$. Пусть $y^*$ — одна из бутстрэп-выборок.

  \begin{enumerate}
    % \item Просто для удобства выпиши $\E(y_i)$, $\Var(y_i)$, $\E(\bar y)$, $\Var(\bar y)$.
    \item Найди $\E(y^*_i)$, $\Var(y^*_i)$, $\E(\bar y^*)$, $\Var(\bar y^*)$.
    \item Найди $\Cov(y_i, y_i^*)$, $\Cov(\bar y, \bar y^*)$.
    \item Что происходит с указанными величинами при $n\to\infty$?
  \end{enumerate}


\item Исследователь Винни-Пух использует две модели, описывающие вектор $y=(y_1, y_2, \ldots, y_n)$. 
Одна модель подсказана Совой, вторая — Кроликом. 
Как известно, у Винни-Пуха опилки в голове, поэтому обе модели содержат $k=0$ параметров.

Величины $y_i$ в обеих моделях и в реальности независимы и одинаково распределены.

Рассмотрим оценку $\hat \Delta = (AIC_{\text{Кролик}} - AIC_{\text{Сова}})/2$ для параметра $\Delta = KL(p||p_{\text{Кролик}}) - KL(p||p_{\text{Сова}})$.

Здесь $p$ — реальное распределение вектора $y$, а $p_{\text{Кролик}}$ и $p_{\text{Сова}}$ — модельные.

\begin{enumerate}
    \item Верно ли, что оценка Винни-Пуха $\hat \Delta$ является несмещённой?
    \item Верно ли, что оценка $\hat \Delta$ является состоятельной? В каком смысле в данном случае корректно говорить о состоятельности?
\end{enumerate}

\item Исследователь Пятачок считает, что в модели 
\[
y_i = \beta_0 + \beta_1 x_i + \varepsilon_i
\]
имеется гетероскедастичность следующего вида: $\Var(\varepsilon_i) = \exp(\alpha_0 + \alpha_1 x_i)$.

\begin{enumerate}
	\item Скорректируйте гетероскедастичность и выведите формулу эффективной оценки в явном виде. 
	\item Поясните, как построить доверительный интервал, устойчивый к гетероскедастичности, используя стандартные ошибки Уайта.
	\item Сформулируйте гипотезу о гомоскедастичности и найдите оценки неизвестных параметров в предположении о гомоскедастичности методом максимального правдоподобия.
\end{enumerate}


\newpage



\item Исследователь Кролик сравнивает два сорта Морковки, А и Б. Морковки сорта А имеют случайный размер $\cN(\mu_A, \sigma^2_A)$.
Морковки сорта Б имеют случайный размер $\cN(\mu_B, \sigma^2_B)$.

У Кролика $n$ наблюдений по морковам сорта А, $a_1$, \ldots, $a_n$, и ещё $n$ наблюдений по морковкам сорта Б, $b_1$, \ldots, $b_n$.
Все четыре параметра — неизвестные.

\begin{enumerate}
    \item Найдите оценки всех параметров с помощью максимального правдоподобия. 
    \item Выведите явную формулу для статистики Вальда для проверки гипотезы $\mu_A = \mu_B$.
\end{enumerate}


\item Исследователь Кролик сравнивает два сорта Морковки, А и Б. Вероятность того, что морковка имеет сорт А описывается функцией
\[
\P(y_i = A \mid r_i ) = \Lambda(\beta_1 + \beta_2 r_i),    
\]
где $r_i$ — размер морковки и $\Lambda$ — логистическая функция.

Выведите явную формулу для статистики множителей Лагранжа для проверки гипотезы $\beta_2 = 0$.



\item Исследователь Кролик знает, что размер сорта морковки А имеет нормальное распределение $\cN(10, 9)$.
Сорт Б Кролику не знаком, поэтому он предполагает, что размер морковки сорта Б имеет нормальное распределение $\cN(\mu, \sigma^2)$.

На поле равновероятно встречаются оба сорта морковки. К сожалению, определить сорт морковки по её виду Кролик не может. 

У Кролика есть 100 наблюдений $r_i$ за размером морковки. 

\begin{enumerate}
    \item Выпишите функцию правдоподобия для данной задачи. 
    \item Выпишите условия первого порядка. Если возможно, найдите оценки правдоподобия в явном виде. 
\end{enumerate}





\end{enumerate}


\end{document}