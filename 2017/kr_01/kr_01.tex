\documentclass[12pt, a4paper]{article}

\usepackage[top=1.5cm, left=2cm, right=2cm, bottom=1.5cm]{geometry} % размер текста на странице

\usepackage{tikz} % картинки в tikz
\usepackage{microtype} % свешивание пунктуации

\usepackage{array} % для столбцов фиксированной ширины

\usepackage{indentfirst} % отступ в первом параграфе

\usepackage{sectsty} % для центрирования названий частей
\allsectionsfont{\centering}

\usepackage{amsmath} % куча стандартных математических плюшек
\usepackage{amssymb} % и символов
\usepackage{bbm}

\usepackage{multicol} % текст в несколько колонок

\usepackage{lastpage} % чтобы узнать номер последней страницы

\usepackage{enumitem} % дополнительные плюшки для списков
%  например \begin{enumerate}[resume] позволяет продолжить нумерацию в новом списке




\usepackage{fontspec} % хз
\usepackage{polyglossia} % для выбора языка в xelatex

\setmainlanguage{russian}
\setotherlanguages{english}

% download "Linux Libertine" fonts:
% http://www.linuxlibertine.org/index.php?id=91&L=1
\setmainfont{Linux Libertine O} % or Helvetica, Arial, Cambria
% why do we need \newfontfamily:
% http://tex.stackexchange.com/questions/91507/
\newfontfamily{\cyrillicfonttt}{Linux Libertine O}

\AddEnumerateCounter{\asbuk}{\russian@alph}{щ} % для списков с русскими буквами
\setlist[enumerate, 2]{label=\asbuk*),ref=\asbuk*} % списки уровня 2 будут буквами а) б) ...

\usepackage{todonotes} % для вставки в документ заметок о том, что осталось сделать
% \todo[inline]{Здесь надо коэффициенты исправить}
% \missingfigure{Здесь будет картина Последний день Помпеи}
% команда \listoftodos — печатает все поставленные \todo'шки

\usepackage{booktabs} % красивые таблицы
% заповеди из документации:
% 1. Не используйте вертикальные линии
% 2. Не используйте двойные линии
% 3. Единицы измерения помещайте в шапку таблицы
% 4. Не сокращайте .1 вместо 0.1
% 5. Повторяющееся значение повторяйте, а не говорите "то же"


% \usepackage[left=1cm,right=1cm,top=1cm,bottom=1cm]{geometry}

\usepackage{fancyhdr} % весёлые колонтитулы
\pagestyle{fancy}
\lhead{Эконометрика, финтех}
\chead{Хэллоуин!}
\rhead{31.10.2017}
\lfoot{}
\cfoot{}
\rfoot{\thepage/\pageref{LastPage}}
\renewcommand{\headrulewidth}{0.4pt}
\renewcommand{\footrulewidth}{0.4pt}

\DeclareMathOperator{\E}{\mathbb{E}}
\let\P\relax
\DeclareMathOperator{\P}{\mathbb{P}}
\DeclareMathOperator{\Var}{\mathbb{V}ar}
\DeclareMathOperator{\Cov}{\mathbb{C}ov}



%% эконометрические сокращения
\def \hb{\hat{\beta}}
\DeclareMathOperator{\sVar}{sVar}
\DeclareMathOperator{\sCov}{sCov}
\DeclareMathOperator{\sCorr}{sCorr}

\def \1{\mathbbm{1}}

\def \hs{\hat{s}}
\def \hy{\hat{y}}
\def \hY{\hat{Y}}
\def \he{\hat{\varepsilon}}
\def \v1{\vec{1}}
\def \cN{\mathcal{N}}
\def \e{\varepsilon}
\def \z{z}

\def \hVar{\widehat{\Var}}
\def \hCorr{\widehat{\Corr}}
\def \hCov{\widehat{\Cov}}

\DeclareMathOperator{\tr}{tr}
\DeclareMathOperator*{\plim}{plim}

%% лаг
\renewcommand{\L}{\mathrm{L}}


\begin{document}


\begin{enumerate}

\item По 100 наблюдениям храбрый Василий под самый Хэллоуин оценил модель $\hy_i = 3.4 - 5.6 x_i + 2.7 z_i$. Стандартная ошибка коэффициента при $x_i$ равна $1.1$.

  \begin{enumerate}
    \item Постройте 95\% доверительный интервал для $\beta_x$.
    \item Проверьте гипотезу $H_0$: $\beta_x = -4$ на уровне значимости 5\%.
  \end{enumerate}

\item Рассмотрим векторы: $x = (2, 0, 1)$, $z = (1, 0, 1)$ и $y= (1, 2, 3)$.
\begin{enumerate}
  \item Найдите матрицу-шляпницу, проецирующую любой вектор на линейную оболочку векторов $x$ и $z$.
  \item Найдите коэффициенты в регрессии $y$ на $x$ и $z$ без константы.
  \item Найдите $ESS$, $RSS$ и $TSS$. Верно ли, что $TSS=ESS+RSS$ в этой модели?
  \item Найдите $\hVar(\hb)$.
\end{enumerate}



\item В рамках классической регрессионной модели $y=X\beta + u$, где $\E(u)=0$, $\Var(u)=\sigma^2 \cdot I$, вектор $\hb$ оценивается с помощью МНК. Обозначим $\hy=X\hb$, $\hat{u}=y-\hy$.

Найдите $\E(\hy)$, $\E(\hat{u})$, $\Var(\hy)$, $\Cov(\hy, \hb)$.

\item Рассмотрим модель парной регрессии $y = \beta_1 \cdot \1 + \beta_2 x + u$.

\begin{enumerate}
  \item Нарисуйте векторы $x$, $\1$, $y$, $\hy$, $\bar y \cdot \1$.
  \item Укажите все прямые углы на рисунке.
  \item Отметьте угол, квадрат косинуса которого равен $R^2$.
  \item Закончите фразу так, чтобы она была корректной
  \begin{enumerate}
    \item Вектор $y - \bar{y}\cdot \1$ — это проекция вектора $y$ на \ldots
    \item Вектор $\hb_2 (x - \bar x\cdot \1)$ — это проекция вектора $y$ на \ldots
  \end{enumerate}
\end{enumerate}

\item Компоненты вектора $z= (z_1, z_2, z_3)$ независимы и имеют экспоненциальное распределение с $\lambda = 1$.

Найдите совместную функцию плотности вектора $y = (z_1z_2z_3, z_1z_2, z_1)$.

\item Верны ли следующие утверждения:

\begin{enumerate}
  \item Сумма двух независимых гамма-распределений с одинаковым $\lambda$ имеет гамма-распределение;
  \item Если помножить $\pi$ на гамма-распределение получится гамма-распределение;
  \item Сумма двух независимых бета-распределений  имеет бета-распределение;
  \item Если помножить $\pi$ на бета-распределение получится бета-распределение;
\end{enumerate}

\item Компоненты вектор вектора $z= (z_1, z_2, z_3)$ независимы и имеют нормальное распределение $z_i \sim \cN(i; 1)$.

Настойчивый исследователь Василий проецирует вектор $z$ на плоскость $z_1 + 2z_2 +3z_3 =0$ и получает вектор $w$. Определим $r = z - (1, 2, 3)$.

\begin{enumerate}
  \item Какое распределение имеет величина $|w|^2$?
  \item Какое распределение имеет $|z - w - (1, 2, 3)|^2$?
  \item Какое распределение имеет $2\cdot |z - w - (1, 2, 3)|^2 / |w|^2$?
\end{enumerate}

\newpage
\item Регрессионная модель имеет вид $y_i= \beta_1+ \beta_x x_{i}+ \beta_z z_{i}+ \beta_w w_{i}+u_i$. Исследователь Феофан оценил эту модель по 20 наблюдениям и оказалось, что $R^2=0.9$. Феофан хочет проверить гипотезу $H_0$ о том, что $\beta_x = \beta_z$ и одновременно $\beta_z + \beta_w = 0$. Предпосылки теоремы Гаусса-Маркова на ошибки $u_i$ выполнены, кроме того, $u_i$ нормально распределены.

\begin{enumerate}
\item Какую вспомогательную регрессию достаточно оценить Феофану для проверки $H_0$?
\item Во вспомогательной регрессии оказалось, что $R^2 = 0.6$. Отвергается ли $H_0$ на 5\%-ом уровне значимости?
\item На сколько процентов изменилась несмещённая оценка дисперсии случайной ошибки при переходе ко вспомогательной регрессии?
\end{enumerate}



\end{enumerate}








\end{document}
