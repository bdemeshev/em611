\documentclass[12pt]{article} % размер шрифта

\usepackage{tikz} % картинки в tikz
\usepackage{microtype} % свешивание пунктуации

\usepackage{array} % для столбцов фиксированной ширины

\usepackage{url} % для вставки ссылок \url{...}

\usepackage{indentfirst} % отступ в первом параграфе

\usepackage{sectsty} % для центрирования названий частей
\allsectionsfont{\centering} % приказываем центрировать все sections

\usepackage{amsthm} % теоремы и доказательства

\theoremstyle{definition} % прямой шрифт в условии теорем
\newtheorem{theorem}{Теорема}[section]


\usepackage{amsmath, amssymb} % куча стандартных математических плюшек

\usepackage[top=2cm, left=1.5cm, right=1.5cm, bottom=2cm]{geometry} % размер текста на странице

\usepackage{lastpage} % чтобы узнать номер последней страницы

\usepackage{enumitem} % дополнительные плюшки для списков
%  например \begin{enumerate}[resume] позволяет продолжить нумерацию в новом списке
\usepackage{caption} % подписи к картинкам без плавающего окружения figure


\usepackage{fancyhdr} % весёлые колонтитулы
\pagestyle{fancy}
\lhead{Эконометрика, финтех}
\chead{}
\rhead{2017-10-28, Праздник-1}
\lfoot{}
\cfoot{}
\rfoot{\thepage/\pageref{LastPage}}
\renewcommand{\headrulewidth}{0.4pt}
\renewcommand{\footrulewidth}{0.4pt}



\usepackage{todonotes} % для вставки в документ заметок о том, что осталось сделать
% \todo{Здесь надо коэффициенты исправить}
% \missingfigure{Здесь будет картина Последний день Помпеи}
% команда \listoftodos — печатает все поставленные \todo'шки

\usepackage{booktabs} % красивые таблицы
% заповеди из документации:
% 1. Не используйте вертикальные линии
% 2. Не используйте двойные линии
% 3. Единицы измерения помещайте в шапку таблицы
% 4. Не сокращайте .1 вместо 0.1
% 5. Повторяющееся значение повторяйте, а не говорите "то же"

\usepackage{fontspec} % поддержка разных шрифтов
\usepackage{polyglossia} % поддержка разных языков

\setmainlanguage{russian}
\setotherlanguages{english}

\setmainfont{Linux Libertine O} % выбираем шрифт
% если Linux Libertine не установлен, то
% можно также попробовать Helvetica, Arial, Cambria и т.Д.

% чтобы использовать шрифт Linux Libertine на личном компе,
% его надо предварительно скачать по ссылке
% http://www.linuxlibertine.org/index.php?id=91&L=1

% на сервисах типа sharelatex.com этот шрифт есть :)

\newfontfamily{\cyrillicfonttt}{Linux Libertine O}
% пояснение зачем нужно шаманство с \newfontfamily
% http://tex.stackexchange.com/questions/91507/

\AddEnumerateCounter{\asbuk}{\russian@alph}{щ} % для списков с русскими буквами
\setlist[enumerate, 2]{label=\asbuk*),ref=\asbuk*} % списки уровня 2 будут буквами а) б) ...

%% эконометрические и вероятностные сокращения
\DeclareMathOperator{\Cov}{Cov}
\DeclareMathOperator{\sCov}{sCov}
\DeclareMathOperator{\sVar}{sVar}

\DeclareMathOperator{\Corr}{Corr}
\DeclareMathOperator{\Var}{Var}
\DeclareMathOperator{\E}{E}
\def \hb{\hat{\beta}}
\def \hs{\hat{\sigma}}
\def \htheta{\hat{\theta}}
\def \s{\sigma}
\def \hy{\hat{y}}
\def \hY{\hat{Y}}
\def \v1{\vec{1}}
\def \e{\varepsilon}
\def \he{\hat{\e}}
\def \z{z}
\def \hVar{\widehat{\Var}}
\def \hCorr{\widehat{\Corr}}
\def \hCov{\widehat{\Cov}}
\def \cN{\mathcal{N}}


\begin{document}

\begin{enumerate}
  \item Найдите SVD-разложение матрицы
  $\begin{pmatrix}
   1 & -1 & 0 & 0 & 0\\
   1 & -1 & 1 & -1 & 1\\
  \end{pmatrix}$.

  \item По 200 наблюдениям эконометресса Аглая оценила две модели. 
  
  Модель А:
  \[
  \hy_i = \underset{(1.2)}{3.4} + \underset{(3.2)}{1.7} x_i, \quad R^2 = 0.8 
  \]
  Модель Б:
  \[
  \hy_i = \underset{(1.7)}{3.1} + \underset{(3.3)}{1.8} x_i  + \underset{(1.9)}{2.1} z_i + \underset{(2.2)}{3.8} w_i, \quad R^2 = 0.9
  \]

  В скобках указаны стандартные ошибки оценок коэффициентов. 

  \begin{enumerate}
    \item Для модели А постройте 99\%-й доверительный интервал для коэффициента $\beta_x$.
    \item Сравните модель А и модель Б с помощью подходящего теста при уровне значимости 1\%.
    Аккуратно сформулируйте $H_0$ и $H_a$.
  \end{enumerate}



  \item Рассмотрим метод главных компонент с центрированием исходных переменных.
  Докажите, что  
  сумма выборочных дисперсий всех главных компонент равна сумме выборочных дисперий всех исходных 
  переменных. 
  
  \item Истинная зависимость имеет вид $y=Z\beta + u$, $\E(u)=0$, $\Var(u) = \sigma^2 \cdot I$. 
  Эконометресса Аглая не знает, что $y$ зависит от $Z$, 
  и, по привычке, строит регрессию на $X$, то есть $\hat y = X\hb$.
  Аглая использует обычный метод наименьших квадратов. 
  
  
  Найдите $\Var(\hb)$, $\E(\hb)$, $\E(RSS)$, $\Var(\hat y)$, $\Cov(\hat u, \hat y)$.
  
  \item Все переменные, и предикторы, и зависимая, центрированы. Рассмотрим задачу гребневой регрессии (ridge regression).
  \[
  Q(\hb) = (y - X\hb)^T (y - X\hb) + \lambda \hb^T \hb \to \min_{\hb}
  \]
  
\begin{enumerate}
  \item Найдите $dQ$ и $d^2Q$;
  \item Найдите явную формулу для $\hb_{ridge}$. Докажите, что это действительно минимум. 
\end{enumerate}

\newpage
\item Вспомнив Матрицу-Мать-Всех-Регрессий, докажите, что
в регрессии
\[
\hat y_i = \hat\beta_1 + \hat \beta_x x_i + \hat \beta_z z_i + \hat \beta_w w_i
\]
величину $R^2$ можно разложить в сумму:
\[
R^2 = \hat \beta_x \frac{\sCov(y, x)}{\sVar(y)} + \hat \beta_z \frac{\sCov(y, z)}{\sVar(y)} + \hat \beta_w \frac{\sCov(y, w)}{\sVar(y)}
\]

Здесь $\sVar$, $\sCov$ — это выборочная дисперсия и выборочная ковариация. 

\item Внедрённый в глубокий тыл противника майор Пронин хочет оценить коэффициенты 
в регрессии $\hat y = \hb_1 + \hb_2 x_i + \hb_3 z_i$. 
Однако, чтобы не привлекать внимания, майор Пронин хочет обойтись построением нескольких 
парных регрессий и применением теоремы Фриша-Во. 

По соображениям секретности майор Пронин не может выполнять векторных и матричных операций, 
а также суммирования $n$ слагаемых. 
Максимум, что майор может незаметно сделать — это посчитать остатки одной регрессии и
использовать их как регрессор в другой регрессии. 

\begin{enumerate}
  \item Сколько парных регрессий ему придётся построить, чтобы оценить коэффициенты $\hb_2$ и $\hb_3$? Какие конкретно?
  \item Возможно ли только с помощью парных регрессий, без дополнительных арифметических действий, оценить $\hb_1$?
\end{enumerate}


\end{enumerate}



\end{document}
