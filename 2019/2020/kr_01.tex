\documentclass[12pt]{article} % размер шрифта

\usepackage{tikz} % картинки в tikz
\usepackage{microtype} % свешивание пунктуации

\usepackage{array} % для столбцов фиксированной ширины

\usepackage{url} % для вставки ссылок \url{...}

\usepackage{indentfirst} % отступ в первом параграфе

\usepackage{sectsty} % для центрирования названий частей
\allsectionsfont{\centering} % приказываем центрировать все sections

\usepackage{amsthm} % теоремы и доказательства

\theoremstyle{definition} % прямой шрифт в условии теорем
\newtheorem{theorem}{Теорема}[section]


\usepackage{amsmath, amssymb} % куча стандартных математических плюшек

\usepackage[top=2cm, left=1.5cm, right=1.5cm, bottom=2cm]{geometry} % размер текста на странице

\usepackage{lastpage} % чтобы узнать номер последней страницы

\usepackage{enumitem} % дополнительные плюшки для списков
%  например \begin{enumerate}[resume] позволяет продолжить нумерацию в новом списке
\usepackage{caption} % подписи к картинкам без плавающего окружения figure


\usepackage{fancyhdr} % весёлые колонтитулы
\pagestyle{fancy}
\lhead{Эконометрика, финтех}
\chead{}
\rhead{2020-10-26, Праздник номер раз}
\lfoot{}
\cfoot{}
\rfoot{\thepage/\pageref{LastPage}}
\renewcommand{\headrulewidth}{0.4pt}
\renewcommand{\footrulewidth}{0.4pt}



\usepackage{todonotes} % для вставки в документ заметок о том, что осталось сделать
% \todo{Здесь надо коэффициенты исправить}
% \missingfigure{Здесь будет картина Последний день Помпеи}
% команда \listoftodos — печатает все поставленные \todo'шки

\usepackage{booktabs} % красивые таблицы
% заповеди из документации:
% 1. Не используйте вертикальные линии
% 2. Не используйте двойные линии
% 3. Единицы измерения помещайте в шапку таблицы
% 4. Не сокращайте .1 вместо 0.1
% 5. Повторяющееся значение повторяйте, а не говорите "то же"

\usepackage{fontspec} % поддержка разных шрифтов
\usepackage{polyglossia} % поддержка разных языков

\setmainlanguage{russian}
\setotherlanguages{english}

\setmainfont{Linux Libertine O} % выбираем шрифт
% если Linux Libertine не установлен, то
% можно также попробовать Helvetica, Arial, Cambria и т.Д.

% чтобы использовать шрифт Linux Libertine на личном компе,
% его надо предварительно скачать по ссылке
% http://www.linuxlibertine.org/index.php?id=91&L=1

% на сервисах типа sharelatex.com этот шрифт есть :)

\newfontfamily{\cyrillicfonttt}{Linux Libertine O}
% пояснение зачем нужно шаманство с \newfontfamily
% http://tex.stackexchange.com/questions/91507/

\AddEnumerateCounter{\asbuk}{\russian@alph}{щ} % для списков с русскими буквами
\setlist[enumerate, 2]{label=\asbuk*),ref=\asbuk*} % списки уровня 2 будут буквами а) б) ...

%% эконометрические и вероятностные сокращения
\DeclareMathOperator{\Cov}{Cov}
\DeclareMathOperator{\sCov}{sCov}
\DeclareMathOperator{\sVar}{sVar}
\DeclareMathOperator{\sCorr}{sCorr}

\DeclareMathOperator{\Corr}{Corr}
\DeclareMathOperator{\Var}{Var}
\DeclareMathOperator{\E}{E}
\def \hb{\hat{\beta}}
\def \hs{\hat{\sigma}}
\def \htheta{\hat{\theta}}
\def \s{\sigma}
\def \hy{\hat{y}}
\def \hY{\hat{Y}}
\def \v1{\vec{1}}
\def \e{\varepsilon}
\def \he{\hat{\e}}
\def \z{z}
\def \hVar{\widehat{\Var}}
\def \hCorr{\widehat{\Corr}}
\def \hCov{\widehat{\Cov}}
\def \cN{\mathcal{N}}


\begin{document}

\begin{enumerate}
\item У исследователя Василия есть мощный генератор многомерного нормального распределения
$\cN(0;\sigma^2 I)$, где $I$ — единичная матрица размера $4\times 4$. 
Других источников случайности в жизни Василия нет. Всё остальное предрешено, а Вася — фаталист!

Василий ничего не знает про $\pi$, $e$ и эти самые функции плотности или распределения.
Однако Василий понимает текстовую аксиоматику нормального распределения. 

\begin{enumerate}
\item Предложите Василию способ равномерно генерировать точки на трехмерной сфере.
\item Предложите Василию способ равномерно генерировать точки на отрезке $[0;1]$.
\item Предложите Василию способ равномерно генерировать точки на квадрате $[0;1]\times [0;1]$.
\end{enumerate}


\item Матрица $X$ состоит из двух блоков, $X = [A \vdots B]$. Обозначим
$H_X = X(X'X)^{-1}X'$ и $H_A = A(A'A)^{-1}A'$. 

\begin{enumerate}
    \item Докажите геометрически без манипуляций с матрицами, что $H_A = H_A H_X$.
    \item Докажите алгебраически без геометрических образов, что $H_A = H_A H_X$.
\end{enumerate}


\item У Василия есть три вектора $a$, $b$ и $c$. 
Все вектора имеют нулевую сумму компонент и единичную длину. 

Василий решает найти такой вектор $v$, также с нулевой суммой компонент и единичный длинной, 
который минимизировал бы сумму квадратов расстояний 

\[
    \|a - x\|^2 + \|b - x\|^2 + \|c - x\|^2
\]

Мария решает найти такой вектор $m$, также с нулевой суммой компонент и единичный длинной,
который максимизировал бы сумму выборочных корреляций

\[
\sCorr^2(a, m) +   \sCorr^2(b, m) +\sCorr^2(c, m) 
\]


Правда ли, что у Василия и Марии получится один и тот же вектор? 
Докажите или опровергните.


\item Рассмотрим модель $y_i = \beta_1 + \beta_2 x_i + u_i$. 
Ошибки $u_i$ независимы и нормальны $\cN(0;1)$.

Мария оценивает модель с помощью МНК. 
Для каждой из ситуаций явно приведите подходящую последовательность $x_i$ 
или докажите, что данная ситуация невозможна.

\begin{enumerate}
    \item Обе оценки сходятся к настоящим параметрам по вероятности: $\hb_1 \to \beta_1$, $\hb_2 \to \beta_2$.
\item Ни одна оценка не сходится к настоящему параметру по вероятности: $\hb_1 \not\to \beta_1$, $\hb_2 \not\to \beta_2$.
\item По вероятности $\hb_1 \to \beta_1$, $\hb_2 \not\to \beta_2$
\item По вероятности $\hb_1 \not\to \beta_1$, $\hb_2 \to \beta_2$
\end{enumerate}


\newpage
\item Выборочная корреляционная матрица трех векторов имеет вид 
\[
\begin{pmatrix}
1 & 0.3 & 0 \\
0.3 & 1 & -0.2 \\
0 & -0.2 & 1 \\
\end{pmatrix}    
\]


Известно, что первые две главные компоненты начинаются так:

\[
pc_1 = (0.02, -0.03, \ldots)    
\]

\[
pc_2 = (-0.01, 0.02, \ldots)    
\]



В этой задаче можно находить собственные числа и собственные вектора с помощью компьютера численно без пояснений.



\begin{enumerate}
    \item Найдите веса, с которыми исходные переменные входят в первые две главные компоненты. 


    \item Восстановите наилучшую аппроксимацию для первых двух наблюдений для всех исходных переменных.

\end{enumerate}


\item Даша хочет проверить, правда ли, что оценка по эконометрике ($y_i)$ одинаково зависит от количества 
часов ($x_i$), потраченных на подготовку, для трёх групп лиц. 

Даша оценила четыре регрессии $y_i = \beta_1 + \beta_2 x_i + u_i$: по всем студентам, по не пьющим кофе, 
по пьющим мало кофе, по пьющим много кофе.

Результаты в таблице:

\begin{tabular}{cccc}
	\toprule
	выборка & уравнение & $RSS$ & наблюдений \\
	все студенты & $\hat y_i = 40 + 2.8x_i$ & 10000 & 750 \\
	без кофе & $\hat y_i = 50 + 3.2x_i$ & 3000 & 200 \\
	мало кофе & $\hat y_i = 37 + 1.8x_i$ & 2000 & 300 \\
    много кофе & $\hat y_i = 45 + 3.4x_i$ & 2000 & 250 \\
    \bottomrule
\end{tabular}


Помогите Даше!


\end{enumerate}


\end{document}