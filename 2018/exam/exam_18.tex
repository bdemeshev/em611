\documentclass[12pt]{article} % размер шрифта

\usepackage{tikz} % картинки в tikz
\usepackage{microtype} % свешивание пунктуации

\usepackage{array} % для столбцов фиксированной ширины

\usepackage{url} % для вставки ссылок \url{...}

\usepackage{indentfirst} % отступ в первом параграфе

\usepackage{sectsty} % для центрирования названий частей
\allsectionsfont{\centering} % приказываем центрировать все sections

\usepackage{amsthm} % теоремы и доказательства

\theoremstyle{definition} % прямой шрифт в условии теорем
\newtheorem{theorem}{Теорема}[section]


\usepackage{amsmath, amssymb} % куча стандартных математических плюшек

\usepackage[top=2cm, left=1.5cm, right=1.5cm, bottom=2cm]{geometry} % размер текста на странице

\usepackage{lastpage} % чтобы узнать номер последней страницы

\usepackage{enumitem} % дополнительные плюшки для списков
%  например \begin{enumerate}[resume] позволяет продолжить нумерацию в новом списке
\usepackage{caption} % подписи к картинкам без плавающего окружения figure


\usepackage{fancyhdr} % весёлые колонтитулы
\pagestyle{fancy}
\lhead{Эконометрика, финтех}
\chead{}
\rhead{2018-12-22, Праздник}
\lfoot{}
\cfoot{}
\rfoot{\thepage/\pageref{LastPage}}
\renewcommand{\headrulewidth}{0.4pt}
\renewcommand{\footrulewidth}{0.4pt}



\usepackage{todonotes} % для вставки в документ заметок о том, что осталось сделать
% \todo{Здесь надо коэффициенты исправить}
% \missingfigure{Здесь будет картина Последний день Помпеи}
% команда \listoftodos — печатает все поставленные \todo'шки

\usepackage{booktabs} % красивые таблицы
% заповеди из документации:
% 1. Не используйте вертикальные линии
% 2. Не используйте двойные линии
% 3. Единицы измерения помещайте в шапку таблицы
% 4. Не сокращайте .1 вместо 0.1
% 5. Повторяющееся значение повторяйте, а не говорите "то же"

\usepackage{fontspec} % поддержка разных шрифтов
\usepackage{polyglossia} % поддержка разных языков

\setmainlanguage{russian}
\setotherlanguages{english}

\setmainfont{Linux Libertine O} % выбираем шрифт
% если Linux Libertine не установлен, то
% можно также попробовать Helvetica, Arial, Cambria и т.Д.

% чтобы использовать шрифт Linux Libertine на личном компе,
% его надо предварительно скачать по ссылке
% http://www.linuxlibertine.org/index.php?id=91&L=1

% на сервисах типа sharelatex.com этот шрифт есть :)

\newfontfamily{\cyrillicfonttt}{Linux Libertine O}
% пояснение зачем нужно шаманство с \newfontfamily
% http://tex.stackexchange.com/questions/91507/

\AddEnumerateCounter{\asbuk}{\russian@alph}{щ} % для списков с русскими буквами
\setlist[enumerate, 2]{label=\asbuk*),ref=\asbuk*} % списки уровня 2 будут буквами а) б) ...

%% эконометрические и вероятностные сокращения
\DeclareMathOperator{\Cov}{Cov}
\DeclareMathOperator{\Corr}{Corr}
\DeclareMathOperator{\Var}{Var}
\DeclareMathOperator{\sCov}{sCov}
\DeclareMathOperator{\sCorr}{sCorr}
\DeclareMathOperator{\sVar}{sVar}
\DeclareMathOperator{\E}{E}
\def \hb{\hat{\beta}}
\def \hs{\hat{\sigma}}
\def \htheta{\hat{\theta}}
\def \s{\sigma}
\def \hy{\hat{y}}
\def \hY{\hat{Y}}
\def \v1{\vec{1}}
\def \e{\varepsilon}
\def \he{\hat{\e}}
\def \z{z}
\def \hVar{\widehat{\Var}}
\def \hCorr{\widehat{\Corr}}
\def \hCov{\widehat{\Cov}}
\def \cN{\mathcal{N}}


\begin{document}

\begin{enumerate}

\item Про законы распределения:

\begin{enumerate}
  \item Василий спроецировал $n$-мерный стандартный нормальный вектор $u$
  на линейную оболочку независимых векторов $a$, $b$ и $c$. Квадрат длины проекции назовём буквой $Z$. Как распределена величина $Z$?
  \item Далее Василий спроецировал $u$ на линейную оболочку векторов $a$ и $b$. Квадрат длины проекции назовём буквой $W$. Как распределена величина $W$?
  \item Неугомонный Василий спроецировал $u$ на линейную оболочку векторов $d$ и $e$. Вектора $d$ и $e$ независимы и ортогональны $a$, $b$ и $c$.
Квадрат длины проекции назовём буквой $Q$. Как распределена величина $Q$?
\item Какое известное распределение можно получить из $Q/Z$? Как конкретно его получить?
\item Какое известное распределение можно получить из $W/Z$? Как конкретно его получить?
\end{enumerate}


\item Рассмотрим модель множественной регрессии, $y=X\beta + u$, где регрессоры детерминистические, а $u \sim \cN(0; \sigma^2 \cdot I)$.
Величина $\sigma^2$ известна. Мы хотим проверить гипотезу $H_0$: $\beta = 0$.


\begin{enumerate}
\item Выведите формулы для статистик $W$, $LR$, $LM$.
\item Сравните эти статистики между собой.
\end{enumerate}



  \item Рассмотрим модель множественной регрессии, $y=X\beta + u$, где регрессоры детерминистические, а $u \sim \cN(0; \sigma^2 \cdot I)$.
  Величина $\sigma^2$ неизвестна и тоже оценивается.
  Мы хотим проверить гипотезу $H_0$: $\beta = 0$.

  \begin{enumerate}
    \item Выведите формулы для статистик $W$, $LR$, $LM$.
    \item Сравните эти статистики между собой.
  \end{enumerate}

  \item Сэр Томас Байес в 18 веке решил задачу, которая на современном языке
  формулируется так:

  Величина $R$ имеет равномерное распределение на отрезке $[0;1]$.
  Мы изготавливаем монетку, выпадающую орлом с вероятностью $R$.
  Затем подбрасываем её $n$ раз. Из этих $n$ раз оказывается $X$ орлов и
  $Y$ решек.

\begin{enumerate}
    \item Как выглядит условная плотность величины $R$ при известных $X$ и $Y$ с точностью до константы?
    \item Какова условная вероятность того, что монетка выпадет орлом, при известных $X$ и $Y$?
\end{enumerate}

  Хинт: какое там есть распределение-то на отрезке $[0;1]$?
  А тут ещё две известных величины, $X$ и $Y$ завалялись :)

  \item Вспомнив Матрицу-Мать-Всех-Регрессий, докажите, что
в регрессии
\[
\hat y_i = \hat\beta_1 + \hat \beta_x x_i + \hat \beta_z z_i + \hat \beta_w w_i
\]
величину $R^2$ можно разложить в сумму:
\[
R^2 = \hat \beta_x \frac{\sCov(y, x)}{\sVar(y)} + \hat \beta_z \frac{\sCov(y, z)}{\sVar(y)} + \hat \beta_w \frac{\sCov(y, w)}{\sVar(y)}
\]


  \item Посмотрим, кто прорешал первую кр :)

  Докажите, что в методе главных компонент с масштабированием переменных средняя величина $R^2$ по всем парным
  регрессиям исходных переменных на первую главную компоненту равна наибольшему сингулярному значению
  матрицы исходных переменных.


\newpage


\item Величины $U_1$ и $U_2$ независимы и равномерны $U[0;1]$. Рассмотрим пару величин $Y_1 = R\cdot \cos \alpha$, $Y_2 = R\cdot \sin \alpha$, где $R=\sqrt{-2\ln U_1}$, а $\alpha = 2\pi U_2$.
\begin{enumerate}
  \item Выпишите дифференциальную форму для пары $U_1$, $U_2$;
  \item Выпишите дифференциальную форму для пары $Y_1$, $Y_2$;
  \item  Найдите совместный закон распределения $Y_1$ и $Y_2$;
  \item Верно ли, что $Y_1$ и $Y_2$ независимы?
  \item  Как распределены $Y_1$ и $Y_2$ по отдельности?

\end{enumerate}

  \item Эта задача посвящена доказательству неравенства Крамера-Рао. Суть его в том, что если мы возьмём любую несмещённую оценку, то её дисперсия будет не меньше некоторой границы. А именно, если $\hat a$ — любая несмещённая оценка вектора $a$, то матрица $M$,
  \[
  M = \Var(s(a))\cdot \Var(\hat a) - I_{k\times k}
  \]
  неотрицательно определена. В этой задаче $\hat a$ — произвольная несмещённая оценка, не обязательно равная $\hat a_{ML}$!
  Как обычно, $s(a)$ — градиент функции правдоподобия в истинной точке.

    \begin{enumerate}
      \item Вспомните, чему равно $\E(s(a))$.
      \item Найдите скаляры $\Cov\left(\hat a_1, \frac{\partial \ell}{\partial a_1}\right)$,
	$\Cov\left(\hat a_1, \frac{\partial \ell}{\partial a_2}\right)$
	и матрицу $\Cov\left(\hat a, s(a) \right)$.
      \item Рассмотрим два произвольных случайных вектора $r$ и $s$ и два вектора констант подходящей длины $\alpha$ и $\beta$.
	Найдите минимум функции $f(\alpha, \beta) = \Var(\alpha^T r + \beta^T s)$ по $\beta$.
	Выпишите явно $\beta^*(\alpha)$ и $f^*(\alpha)$.
      \item Докажите, что для произвольных случайных векторов положительно определена матрица
	\[
          \Var(r) - \Cov(r, s) \Var^{-1}(s)\Cov(s, r)
	\]
      \item Завершите доказательство векторного неравенства Крамера-Рао.
    \end{enumerate}




\end{enumerate}



\end{document}
