\documentclass[12pt]{article} % размер шрифта
\usepackage{tikz} % картинки в tikz
\usepackage{graphicx}
\usepackage{amssymb}
\graphicspath{{images/}}
\usepackage{microtype} % свешивание пунктуации
\usepackage{array} % для столбцов фиксированной ширины
\usepackage{url} % для вставки ссылок \url{...}
\usepackage{indentfirst} % отступ в первом параграфе
\usepackage{sectsty} % для центрирования названий частей
\allsectionsfont{\centering} % приказываем центрировать все sections
\usepackage{amsthm} % теоремы и доказательства
\theoremstyle{definition} % прямой шрифт в условии теорем
\newtheorem{theorem}{Теорема}[section]
\usepackage{amsmath} % куча стандартных математических плюшек
\usepackage[top=2cm, left=1.5cm, right=1.5cm, bottom=2cm]{geometry} % размер текста на странице
\usepackage{lastpage} % чтобы узнать номер последней страницы
\usepackage{enumitem} % дополнительные плюшки для списков
%  например \begin{enumerate}[resume] позволяет продолжить нумерацию в новом списке
\usepackage{caption} % подписи к картинкам без плавающего окружения figure
\usepackage{hyperref}


\usepackage{fancyhdr} % весёлые колонтитулы
\pagestyle{fancy}
\lhead{Эконометрика, финтех}
\chead{}
\rhead{2018-12-08, встреча 9}
\lfoot{}
\cfoot{}
\rfoot{\thepage/\pageref{LastPage}}
\renewcommand{\headrulewidth}{0.4pt}
\renewcommand{\footrulewidth}{0.4pt}



\usepackage{todonotes} % для вставки в документ заметок о том, что осталось сделать
% \todo{Здесь надо коэффициенты исправить}
% \missingfigure{Здесь будет картина Последний день Помпеи}
% команда \listoftodos — печатает все поставленные \todo'шки

\usepackage{booktabs} % красивые таблицы
% заповеди из документации:
% 1. Не используйте вертикальные линии
% 2. Не используйте двойные линии
% 3. Единицы измерения помещайте в шапку таблицы
% 4. Не сокращайте .1 вместо 0.1
% 5. Повторяющееся значение повторяйте, а не говорите "то же"

\usepackage{fontspec} % поддержка разных шрифтов
\usepackage{polyglossia} % поддержка разных языков

\setmainlanguage{russian}
\setotherlanguages{english}

\setmainfont{Linux Libertine O} % выбираем шрифт
% если Linux Libertine не установлен, то
% можно также попробовать Helvetica, Arial, Cambria и т.Д.

% чтобы использовать шрифт Linux Libertine на личном компе,
% его надо предварительно скачать по ссылке
% http://www.linuxlibertine.org/index.php?id=91&L=1

% на сервисах типа sharelatex.com этот шрифт есть :)

\newfontfamily{\cyrillicfonttt}{Linux Libertine O}
% пояснение зачем нужно шаманство с \newfontfamily
% http://tex.stackexchange.com/questions/91507/

\AddEnumerateCounter{\asbuk}{\russian@alph}{щ} % для списков с русскими буквами
\setlist[enumerate, 2]{label=\asbuk*),ref=\asbuk*} % списки уровня 2 будут буквами а) б) ...

%% эконометрические и вероятностные сокращения
\DeclareMathOperator{\Cov}{Cov}
\DeclareMathOperator{\sCov}{sCov}
\DeclareMathOperator{\sVar}{sVar}
\DeclareMathOperator{\sCorr}{sCorr}
\DeclareMathOperator{\Corr}{Corr}
\DeclareMathOperator{\Var}{Var}
\DeclareMathOperator{\E}{E}
\DeclareMathOperator{\tr}{trace}
\DeclareMathOperator{\trace}{trace}
\DeclareMathOperator{\Lin}{Lin}
\DeclareMathOperator{\Linp}{Lin^{\perp}}
\DeclareMathOperator{\Col}{Col}
\DeclareMathOperator{\Colp}{Col^{\perp}}



\def \hb{\hat{\beta}}
\def \hs{\hat{\sigma}}
\def \htheta{\hat{\theta}}
\def \s{\sigma}
\def \hy{\hat{y}}
\def \hY{\hat{Y}}
\def \v1{\vec{1}}
\def \e{\varepsilon}
\def \he{\hat{\e}}
\def \z{z}
\def \hVar{\widehat{\Var}}
\def \hCorr{\widehat{\Corr}}
\def \hCov{\widehat{\Cov}}
\def \cN{\mathcal{N}}
\def \RR{\mathbb{R}}
\def \hu{\hat{u}}
\def \ha{\hat{\alpha}}
\def \hx{\hat{x}}
\def \hg{\hat{\gamma}}


\def \tx{\tilde{x}}
\def \cx{\check{x}}

\def \cF{\mathcal{F}}
\def \cChi{\mathcal{\chi}}


\makeatletter
\def\MT@warn@unknown{}
\makeatother



\begin{document}
Конспектировали: Любовь Корж и Кирилл
Пак
\section{Мирок проекций}
    Случайный вектор $u$ имеет многомерное нормальное стандартное распределение

\begin{center}
\begin{tikzpicture}[x=0.75pt,y=0.75pt,yscale=-1,xscale=1]
%uncomment if require: \path (0,417); %set diagram left start at 0, and has height of 417

%Straight Lines [id:da687435951622533]
\draw    (126.55,302.81) -- (320.47,138.29) ;
\draw [shift={(322,137)}, rotate = 499.69] [color={rgb, 255:red, 0; green, 0; blue, 0 }  ][line width=0.75]    (10.93,-3.29) .. controls (6.95,-1.4) and (3.31,-0.3) .. (0,0) .. controls (3.31,0.3) and (6.95,1.4) .. (10.93,3.29)   ;

%Straight Lines [id:da1949404199859719]
\draw    (126.55,302.81) -- (318,299.04) ;
\draw [shift={(320,299)}, rotate = 538.87] [color={rgb, 255:red, 0; green, 0; blue, 0 }  ][line width=0.75]    (10.93,-3.29) .. controls (6.95,-1.4) and (3.31,-0.3) .. (0,0) .. controls (3.31,0.3) and (6.95,1.4) .. (10.93,3.29)   ;

%Curve Lines [id:da2608125844837438]
\draw    (305,221) .. controls (400,180) and (415,196) .. (420,215) .. controls (425,234) and (411,230) .. (443,229) .. controls (475,228) and (457,254) .. (453,266) .. controls (449,278) and (480,262) .. (488,304) .. controls (496,346) and (170.55,317.81) .. (126.55,302.81) ;


%Straight Lines [id:da10361993069929565]
\draw    (126.55,302.81) -- (305,221) ;


%Straight Lines [id:da019404884624576013]
\draw [color={rgb, 255:red, 0; green, 0; blue, 0 }  ,draw opacity=1 ] [dash pattern={on 0.84pt off 2.51pt}]  (322,137) -- (320,299) ;


%Curve Lines [id:da11010057931623596]
\draw    (126.55,302.81) .. controls (131.74,290.49) and (212,39) .. (174,66) .. controls (136,93) and (140.64,63.42) .. (129.12,56.83) .. controls (117.61,50.24) and (106.91,62.21) .. (106.52,80.46) .. controls (106.14,98.71) and (89.98,62.64) .. (74.83,76.63) .. controls (59.69,90.62) and (71,125) .. (78,143) ;


%Straight Lines [id:da7445976738703185]
\draw    (126.55,302.81) -- (128.97,141) ;
\draw [shift={(129,139)}, rotate = 450.86] [color={rgb, 255:red, 0; green, 0; blue, 0 }  ][line width=0.75]    (10.93,-3.29) .. controls (6.95,-1.4) and (3.31,-0.3) .. (0,0) .. controls (3.31,0.3) and (6.95,1.4) .. (10.93,3.29)   ;

%Straight Lines [id:da8449751294159009]
\draw [color={rgb, 255:red, 0; green, 0; blue, 0 }  ,draw opacity=1 ] [dash pattern={on 0.84pt off 2.51pt}]  (322,137) -- (129,139) ;


%Straight Lines [id:da8847006327048359]
\draw    (126.55,302.81) -- (78,143) ;



% Text Node
\draw (297.61,111.13) node [scale=1]  {$u\ \sim \ N( 0,I) \ $};
% Text Node
\draw (429.05,303.88) node [color={rgb, 255:red, 0; green, 0; blue, 0 }  ,opacity=1 ,rotate=-359.87]  {$\dim V=a$};
% Text Node
\draw (297.35,282.96) node   {$\hat{u}_{V}$};
% Text Node
\draw (122.05,100.88) node [color={rgb, 255:red, 0; green, 0; blue, 0 }  ,opacity=1 ,rotate=-359.87]  {$\dim W=b$};
% Text Node
\draw (121,325) node   {$W\perp V$};
% Text Node
\draw (149.35,153.96) node   {$\hat{u}_{W}$};
\end{tikzpicture}
\end{center}

    Такой вектор удовлетворяет ряду аксиом.
    \par
    \textbf{Аксиомы:}
    \begin{enumerate}
        \item Закон распределения не зависит от угла (а только от длины)
        \item $u_i$ независимы
        \item $\Var(u_i) = 1$
    \end{enumerate}
    Из этих аксиом следует два важных текстовых свойства нормального распределения.
    \par
    \textbf{Свойства:}
    \begin{enumerate}
    \item ${\lVert \hu_V \rVert ^2} \sim\cChi_{a}^2$
    \item Если $ V \perp W $, то $ {\lVert \hu_V \rVert ^2} $ и  $ {\lVert \hu_W \rVert ^2} $ независимы
    \end{enumerate}

    \[
        \cfrac{\lVert \hu_V \rVert ^2 / a}{\lVert \hu_W \rVert ^2 / b}  \sim \cF_{a,b}
    \]
\newpage
\section{Мирок потока происшествий}
    \textbf{Предпосылки:}
    \begin{enumerate}
        \item Количество происшествий за разные интервалы времени независимы
        \item Вероятность того, что за малый интервал времени произойдёт ровно одно (хотя бы одно) событие, пропорциональна длине интервала
        \[
             P(\text{событие в интервале } [t; t + \Delta]) = \lambda\Delta + o(\Delta)
        \]
    \end{enumerate}

    \textbf{Следствия:}
    \begin{enumerate}
        \item Время между соседними событиями имеет экспоненциальное распределение $x_i \sim exp(\lambda)$
        \[
            f(x) = \lambda\exp(-\lambda x)
        \]
        \item Время на $n$ происшествий $x_1 + x_2 + ... + x_n = S_n \sim Gamma(n, \lambda )$
        \[
            f(s) = const \cdot s^{n-1} e^{-\lambda s}
        \]
        \[
            \E(S_n) = n \cdot \frac{1}{\lambda}
        \]
        \[
            \Var(S_n) = n \cdot \Var(X_1) = n \cdot \frac{1}{\lambda ^2}
        \]
        \item Вспомним задачку про тётю Мотю, пекущую блинчики внуку и на продажу

\begin{center}
\begin{tikzpicture}[x=0.75pt,y=0.75pt,yscale=-1,xscale=1]
%uncomment if require: \path (0,300); %set diagram left start at 0, and has height of 300

%Straight Lines [id:da494707727316015]
\draw    (100,109) -- (419,109) ;


\draw   (125.3,102) -- (141.11,117.21)(140.9,101.6) -- (125.51,117.61) ;
\draw   (240.3,101) -- (256.11,116.21)(255.9,100.6) -- (240.51,116.61) ;
\draw   (180.3,101) -- (196.11,116.21)(195.9,100.6) -- (180.51,116.61) ;
\draw   (152.3,103) -- (168.11,118.21)(167.9,102.6) -- (152.51,118.61) ;
\draw   (328.3,101) -- (344.11,116.21)(343.9,100.6) -- (328.51,116.61) ;
\draw   (299.3,101) -- (315.11,116.21)(314.9,100.6) -- (299.51,116.61) ;
\draw   (268.3,101) -- (284.11,116.21)(283.9,100.6) -- (268.51,116.61) ;
%Straight Lines [id:da3951827377002568]
\draw    (112,99) -- (112,121) ;


%Straight Lines [id:da5266324905725944]
\draw    (362,99) -- (362,121) ;


%Straight Lines [id:da9881087289346389]
\draw    (220,99) -- (220,121) ;


%Shape: Brace [id:dp46491745479370816]
\draw   (113,132) .. controls (113.05,136.67) and (115.4,138.98) .. (120.07,138.93) -- (156.57,138.59) .. controls (163.24,138.53) and (166.59,140.83) .. (166.63,145.5) .. controls (166.59,140.83) and (169.9,138.47) .. (176.57,138.4)(173.57,138.43) -- (213.07,138.06) .. controls (217.74,138.01) and (220.05,135.66) .. (220,130.99) ;
%Shape: Brace [id:dp38046343659455084]
\draw   (222,132) .. controls (222.03,136.67) and (224.38,138.98) .. (229.05,138.95) -- (282.05,138.57) .. controls (288.72,138.52) and (292.07,140.83) .. (292.1,145.5) .. controls (292.07,140.83) and (295.38,138.48) .. (302.05,138.43)(299.05,138.45) -- (355.05,138.05) .. controls (359.72,138.02) and (362.03,135.67) .. (362,131) ;

% Text Node
\draw (167,157) node  [align=left] {a штук внуку};
% Text Node
\draw (293,158) node  [align=left] {b штук на продажу};


\end{tikzpicture}
\end{center}
        \[
            A = \frac{S_a}{S_{a+b}} - \text{доля времени на блинчики внуку от общего времени}
        \]
        \[
            A \in [0;1]
        \]
        \[
            \E[A] = \frac{a}{a+b}
        \]
        \[
             \frac{S_a}{S_{a+b}} \sim Beta(a, b) \text{ (по определению)}
        \]
    \end{enumerate}
    \textbf{Частный случай:}\\
    $a = 2$ \text{(внуку)} \\
    $b = 1$ \text{(на продажу)}\\
    Время на три блинчика соответственно равно $x_1, x_2, x_3$
    \[
        A_1 = \frac{X_1}{X_1 + X_2} \sim Beta(1, 1)
    \]
    \[
        A_2 = \frac{X_1 + X_2}{X_1 + X_2 + X_3} \sim Beta(2, 1)
    \]
    \[
        S_3 = X_1 + X_2 + X_3 \sim Gamma(3, \lambda)
    \]
    Выразим $x_1, x_2, x_3$ через $a_1, a_2, s_3$:
    \[
        x_1 = a_1 \cdot a_2 \cdot s_3
    \]
    \[
        x_2 = (1 - a_1) \cdot a_2 \cdot s_3
    \]
    \[
        x_3 = (1 - a_1) \cdot s_3 = s_3 - a_2 \cdot s_3
    \]
    Посчитаем $f(x_1, x_2, x_3) \cdot dx_1 \wedge dx_2 \wedge dx_3$ \\
    Отдельно вычислим $dx_1 \wedge dx_2 \wedge dx_3$:
    \begin{eqnarray*}
        dx_1 \wedge dx_2 = d(a_1 a_2 s_3) \wedge d((1 - a_1) a_2 s_3) = \\
        = d(a_1 a_2 s_3 \wedge (d(a_2 s_3) - d(a_1 a_2 s_3)) = \\
        =d(a_1 a_2 s_3) \wedge d(a_2 s_3) = (da_1 a_2 s_3 + a_1 d(a_2 s_3) \wedge d(a_2 s_3) = \\
        a_2 s_3 da_1 \wedge d(a_2 s_3)
    \end{eqnarray*}
    \begin{eqnarray*}
        dx_1 \wedge dx_2 \wedge dx_3 = a_2 s_3 da_1 \wedge d(a_2 s_3) \wedge d(s_3 - a_2 s_3) = \\
        = a_2 s_3 da_1 \wedge (da_2 s_3 + a_2 ds_3) \wedge ds_3 = \\
        = a_2 s_3^2 \cdot da_1 \wedge da_2 \wedge ds_3
    \end{eqnarray*}
    \[
        \underbrace{\lambda^3 \cdot e^{- \lambda s_3}\cdot a_2 s_3^2}_{\text{совместная ф-ия плотности}} da_1 \wedge da_2 \wedge ds_3
    \]

    \[
       \underbrace{?}_{f(a_1)}  \cdot  \underbrace{? \cdot a_2}_{f(a_2)} \cdot  \underbrace{? \cdot s_3^2 e^{- \lambda s_3}}_{f(s_3)}
    \]
    \[
        1 \cdot 2a_2 \cdot \frac{\lambda^3}{2} s_3^2 e^{- \lambda s_3}
    \]

    \[
        dx_1 \wedge dx_2 \wedge dx_3 = a_2 s_3^2 da_1 \wedge da_2 \wedge ds_3
    \]
    Если печь внуку 3, в не 2 блинчика, то добавится $A3 = \frac{X_1 + X_2 + X_3}{X_1 + X_2 + X_3 + X_4}$,
     $s_3$ превратится в $a_3 s_4$ и $x_4$ в $s_4 - s_3$
    \begin{eqnarray*}
        dx_1 \wedge dx_2 \wedge dx_3 \wedge dx_4 = a_2 s_3^2 da_1 \wedge da_2 \wedge ds_3 \wedge d(s_4 - s_3) = \\
        = a_2 s_3^2 da_1 \wedge da_2 \wedge ds_3 \wedge ds_4 = \\
        = a_2(a_3 s_4)^2 da_1 \wedge da_2 \wedge d(a_3 s_4) \wedge ds_4 = \\
        = a_2 a_3^2 s_4^2 da_1 \wedge da_2 \wedge (da_3 s_4 + ds_4 a_3) \wedge ds_4 = \\
        = a_2 a_3^2 s_4^2 da_1 \wedge da_2 \wedge s_4 da_3 \wedge ds_4 = \\
        = a_2 \cdot a_3^2 \cdot s_4^3 \wedge da_1 \wedge da_2 \wedge da_3 \wedge ds_4
    \end{eqnarray*}
    Получается, что при изготовлении $a$ блинчиков внуку и 1 на продажу
    \[
        f(x) = ? x^{a-1}
    \]
    Аналогично, при изготовлении 1 блинчика внуку и $b$ на продажу
    \[
        f(x) = ? (1 - x)^{b-1}
    \]
    Логично предположить, что при изготовлении $a$ блинчиков внуку и $b$ блинчиков на подажу
    \[
        f(x) = ? x^{a-1} (1 - x)^{b-1}
    \]
    \textbf{Мораль:}\\
    Если $X$ --- доля времени, которое тратится на испечение $a$ блинчиков, если всего пеку $a + b$ блинчиков, то $x \in [0;1)$, $X \sim Beta(a, b)$ и $f_{a,b}(x) = ? x^{a-1} (1 - x)^{b-1}$


    \section{Вывод функции плотности $\cChi^2$-распределения. Связь
    $\cChi^2$ и гамма распределений}

Рассмотрим стандартную нормальную случайную величину $X \sim \cN(0, 1)$. Мы знаем, как выглядит ее функция распределения:
    \[
        f(x)=\frac{1}{\sqrt{2\pi}}e^{-\frac{x^2}{2}}
    \]

Давайте выведем функцию плотности случайной величины, которая имеет $\chi^2$-распределение с 1 степенью свободы.
По определению случайная величина $Q=X^2 \sim \cChi_1^2$ и представляет собой квадрат длины вектора из одной компоненты. Таким образом, нам надо найти функцию плотности распредения случайной величины $Q$.

Воспользуемся дифференциальной формой X (техника с "птичками"):
    \[
        f(x)dx=\frac{1}{\sqrt{2\pi}}e^{-\frac{x^2}{2}}dx
    \]

Сделаем замену. Так как $X=\pm \sqrt{Q}$, то $x=\pm \sqrt{q}$. Заметим, что для получения дифференциальной формы для $Q$ надо подставить в дифференциальную форму для $X$ две точки $x= \sqrt{q}$ и $x=- \sqrt{q}$, что в силу симметричности функции плотности распредения $X$ означает просто домножение на 2:
    \[
        2f(\sqrt{q})d\sqrt{q} = 2 \frac{1}{\sqrt{2\pi}} e^{-\frac{q}{2}} \frac{1}{2 \sqrt{q}} dq = \frac{1}{\sqrt{2\pi}} q^{-\frac{1}{2}} e^{-\frac{q}{2}} dq
    \]

То, что стоит перед дифференциалом, и есть функция плотности случайной величины $Q \sim \cChi_1^2$:
    \[
         f(q) = \frac{1}{\sqrt{2\pi}} q^{-\frac{1}{2}} e^{-\frac{q}{2}}
    \]

Вспомним, что функция плотности гамма-распределения имеет вид:
    \[
         f(s) = const \times s^{n-1} e^{-\lambda s}
    \]

Таким образом, мы с чистой совестью заключаем, что $\chi^2$-распределение с 1 степенью свободы, --- это ни что иное, как гамма-распредение с параметрами $n=\frac{1}{2}$ и $\lambda=\frac{1}{2}$:
    \[
        \cChi_1^2 \sim Gamma\left(n=\frac{1}{2}, \lambda=\frac{1}{2}\right)
    \]

Получается, что $\chi^2$-распределение --- это частный случай гамма-распределения. Например, случайная величина, имеющая $\chi^2$-распределение с 2 степенями свободы, представляет собой сумму двух случайных величин, имеющих $\chi^2$-распределение с 1 степенью свободы, и значит имеет гамма-распредение с параметрами $n=1$ и $\lambda=\frac{1}{2}$:
    \begin{eqnarray*}
        \cChi_2^2 \sim \cChi_1^2 + \cChi_1^2 \sim
        Gamma\left(n=\frac{1}{2}, \lambda=\frac{1}{2}\right) +
        Gamma\left(n=\frac{1}{2}, \lambda=\frac{1}{2}\right) \sim \\
        \sim Gamma\left(n=1, \lambda=\frac{1}{2}\right) \sim
        Exp\left(\frac{1}{2}\right)
    \end{eqnarray*}
Т.к. гамма-распределение с $n=1$ --- это экспоненциальное распределение, то в итоге $\chi^2$-распределение с 2 степенями свободы можно трактовать (в терминах тети Моти, которая печет блинчики), как время ожидания одного блинчика, если в среднем 1 блинчик печется раз в 2 минуты (или, например, 2 часа, в зависимости от единиц измерения времени).

% Revisiting Gauss-Markov
\section{Вспоминая теорему Гаусса-Маркова...}

Теперь мы можем сформулировать несколько теорем.

% Первая теорема
\newtheorem*{theo_n}{Теорема 1}
\begin{theo_n}

    Если:
    % Маркированный список
    \begin{itemize}
        \item \text{[ТГМ1] } Предполагаем, что верна модель $y=X\beta+u$.
        \item \text{[ТГМ2] } Оцениваем модель регрессии $\hy = X\hb$ с помощью МНК.
        \item \text{[ТГМ3] } $\beta$ --- константы.
        \item \text{[ТГМ4] } X стохастистические $(n>k)$:
            \[
                P(X \text{ имеет полный ранг}) = 1
            \]
        \item \text{[ТГМ5] } $\E(u|X)=0$, $\Var(u|X)=\sigma^2I$
        \item \text{[нормальность] }  $u|X \sim \cN(0, \sigma^2I)$
    \end{itemize}
    %/Маркированный список

    то:
    % Маркированный список
    \begin{itemize}
        \item $\cfrac{RSS}{\sigma^2} \sim \cChi_{n-k}^2 \sim Gamma\left(\cfrac{n-k}{2}, \lambda=\cfrac{1}{2}\right)$ \\
        (новизна: гамма-распределение)
    \end{itemize}
    %/Маркированный список

\end{theo_n}
%/Первая теорема

% Вторая теорема
\newtheorem*{theo_nn}{Теорема 2}
\begin{theo_nn}

    Если:
    % Маркированный список
    \begin{itemize}
        \item \text{[ТГМ1'] } Предполагаем, что верна модель $y=\beta_1 \bf{1} +u$.
        \item \text{[ТГМ2] } По-прежнему оцениваем модель регрессии $\hy = X\hb$ с помощью МНК.
        \item Выполнены предпосылки \text{[ТГМ3-5] } и \text{[нормальность] }
    \end{itemize}
    %/Маркированный список

    то:
    % Маркированный список
    \begin{itemize}
        \item $\cfrac{RSS}{\sigma^2} \sim \cChi_{n-k}^2 \sim Gamma\left(\cfrac{n-k}{2}, \lambda=\cfrac{1}{2}\right)$ \\
        (как и было) \\
        но в дополнение:
        \item $\cfrac{ESS}{\sigma^2} \sim \cChi_{k-1}^2 \sim Gamma\left(\cfrac{k-1}{2}, \lambda=\cfrac{1}{2}\right)$
        \item $\cfrac{ESS/(k - 1)}{RSS/(n-k)} \sim \cF_{k - 1, n-k}$ \\
        (позволяет проверить гипотезу: $H_0: \beta_2 = \beta_3 =...=\beta_k=0$)
        \item $\cfrac{TSS}{\sigma^2} \sim \cChi_{n-1}^2 \sim Gamma\left(\cfrac{n-1}{2}, \lambda=\cfrac{1}{2}\right)$
        \item $R^2 = \cfrac{ESS}{TSS} = \cfrac{ESS}{ESS+RSS} \sim Beta\left(\cfrac{k-1}{2}, \cfrac{n-k}{2}\right)$ \\
        (выполнено в силу того, что $\cfrac{ESS}{\sigma^2}$ и $\cfrac{RSS}{\sigma^2}$ независимые случайные величины, имеющие гамма-распределение)
    \end{itemize}
    %/Маркированный список

\end{theo_nn}
%/Вторая теорема

Знание о том, что $R^2$ имеет бета-распределение позволяет, например, найти его математическое ожидание:
    \[
        \E(R^2) = \cfrac{(\frac{k-1}{2})}{\frac{k-1}{2}+\frac{n-k}{2}} = \cfrac{k-1}{n-k}
    \]

% Третья теорема
\newtheorem*{theo_nnn}{Теорема 3}
\begin{theo_nnn}
    Если:
    % Маркированный список
    \begin{itemize}
        \item \text{[ТГМ1''] } Предполагаем, что верна модель
        $y = X\beta + u$, но часть коэффициентов занулена:
        \[
            \beta =
                \begin{pmatrix}
                    \beta_{R+1}  \\ \vdots \\ \beta_{R} \\
                    \beta_{R+1}  \\ \vdots \\ \beta_{UR} \\
                \end{pmatrix}
        \]
        Предполагаем $H_0: \beta_{R+1} = \beta_{R+2} =...=\beta_{UR}=0$
        \item \text{[ТГМ2''] } Оцениваем 2 модели регрессии: короткую (R, в предположении $H_0$) и длинную (UR, $\hy = X\hb$, без ограничений) с помощью МНК. В результате получаем блок показателей по длинной модели ($\hy^{UR}, \hb^{UR}, RSS_{UR}, ESS_{UR}$) и по короткой модели ($\hy^{R}, \hb^{R}, RSS_{R}, ESS_{R}$). TSS у них общий, т.к. не зависит от выбранной модели.
        \item Выполнены предпосылки \text{[ТГМ3-5] } и \text{[нормальность] }
    \end{itemize}
    %/Маркированный список

    то:
    % Маркированный список
    \begin{itemize}
        \item $\cfrac{(RSS_{R} - RSS_{UR})/k_{UR} - k_{R})}{RSS_{UR}/(n-k_{UR})} \sim \cF_{k_{UR} - k_{R}, n-k_{UR}}$ \\
        (в силу независимости $RSS_{UR} \text{ и } RSS_{R} - RSS_{UR}$)
        \item $\cfrac{RSS_{UR}}{RSS_{R}} = \cfrac{RSS_{UR}}{RSS_{UR}+(RSS_{R}-RSS_{UR})} \sim Beta\left(\cfrac{n-k_{UR}}{2}, \cfrac{k_{UR} - k_{R}}{2}\right)$ \\
        (Заметим, что $RSS_{UR} \text{ и } RSS_{R}$ зависимы, а вот $RSS_{UR} \text{ и } RSS_{R}-RSS_{UR}$ независимы)
        \item $\cfrac{RSS_{R}-RSS_{UR}}{RSS_{R}} \sim Beta\left(\cfrac{k_{UR} - k_{R}}{2}, \cfrac{n-k_{UR}}{2}\right)$ \\
        (Заметим, что $RSS_{R}-RSS_{UR} \text{ и } RSS_{R}$ зависимы)
        \item $1-R^2 = \cfrac{RSS}{RSS+ESS} \sim Beta\left(\cfrac{n-k}{2}, \cfrac{k-1}{2}\right)$
    \end{itemize}
    %/Маркированный список


\end{theo_nnn}
\end{document}
