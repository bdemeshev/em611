\documentclass[12pt]{article} % размер шрифта

\usepackage{tikz} % картинки в tikz
\usepackage{microtype} % свешивание пунктуации

\usepackage{array} % для столбцов фиксированной ширины

\usepackage{url} % для вставки ссылок \url{...}

\usepackage{indentfirst} % отступ в первом параграфе

\usepackage{sectsty} % для центрирования названий частей
\allsectionsfont{\centering} % приказываем центрировать все sections

\usepackage{amsthm} % теоремы и доказательства

\theoremstyle{definition} % прямой шрифт в условии теорем
\newtheorem{theorem}{Теорема}[section]


\usepackage{amsmath, amssymb} % куча стандартных математических плюшек

\usepackage[top=2cm, left=1.5cm, right=1.5cm, bottom=2cm]{geometry} % размер текста на странице

\usepackage{lastpage} % чтобы узнать номер последней страницы

\usepackage{enumitem} % дополнительные плюшки для списков
%  например \begin{enumerate}[resume] позволяет продолжить нумерацию в новом списке
\usepackage{caption} % подписи к картинкам без плавающего окружения figure


\usepackage{fancyhdr} % весёлые колонтитулы
\pagestyle{fancy}
\lhead{Эконометрика, финтех}
\chead{}
\rhead{2017-10-06, встреча 3}
\lfoot{}
\cfoot{}
\rfoot{\thepage/\pageref{LastPage}}
\renewcommand{\headrulewidth}{0.4pt}
\renewcommand{\footrulewidth}{0.4pt}



\usepackage{todonotes} % для вставки в документ заметок о том, что осталось сделать
% \todo{Здесь надо коэффициенты исправить}
% \missingfigure{Здесь будет картина Последний день Помпеи}
% команда \listoftodos — печатает все поставленные \todo'шки

\usepackage{booktabs} % красивые таблицы
% заповеди из документации:
% 1. Не используйте вертикальные линии
% 2. Не используйте двойные линии
% 3. Единицы измерения помещайте в шапку таблицы
% 4. Не сокращайте .1 вместо 0.1
% 5. Повторяющееся значение повторяйте, а не говорите "то же"

\usepackage{fontspec} % поддержка разных шрифтов
\usepackage{polyglossia} % поддержка разных языков

\setmainlanguage{russian}
\setotherlanguages{english}

\setmainfont{Linux Libertine O} % выбираем шрифт
% если Linux Libertine не установлен, то
% можно также попробовать Helvetica, Arial, Cambria и т.Д.

% чтобы использовать шрифт Linux Libertine на личном компе,
% его надо предварительно скачать по ссылке
% http://www.linuxlibertine.org/index.php?id=91&L=1

% на сервисах типа sharelatex.com этот шрифт есть :)

\newfontfamily{\cyrillicfonttt}{Linux Libertine O}
% пояснение зачем нужно шаманство с \newfontfamily
% http://tex.stackexchange.com/questions/91507/

\AddEnumerateCounter{\asbuk}{\russian@alph}{щ} % для списков с русскими буквами
\setlist[enumerate, 2]{label=\asbuk\cdot),ref=\asbuk\cdot} % списки уровня 2 будут буквами а) б) ...

%% эконометрические и вероятностные сокращения
\DeclareMathOperator{\Cov}{Cov}
\DeclareMathOperator{\sCov}{sCov}
\DeclareMathOperator{\sVar}{sVar}
\DeclareMathOperator{\sCorr}{sCorr}
\DeclareMathOperator{\Corr}{Corr}
\DeclareMathOperator{\Var}{Var}
\DeclareMathOperator{\E}{E}
\DeclareMathOperator{\tr}{trace}
\DeclareMathOperator{\trace}{trace}
\DeclareMathOperator{\Lin}{Lin}
\DeclareMathOperator{\Linp}{Lin^{\perp}}


\def \hb{\hat{\beta}}
\def \hs{\hat{\sigma}}
\def \htheta{\hat{\theta}}
\def \s{\sigma}
\def \hy{\hat{y}}
\def \hY{\hat{Y}}
\def \v1{\vec{1}}
\def \e{\varepsilon}
\def \he{\hat{\e}}
\def \z{z}
\def \hVar{\widehat{\Var}}
\def \hCorr{\widehat{\Corr}}
\def \hCov{\widehat{\Cov}}
\def \cN{\mathcal{N}}
\def \RR{\mathbb{R}}


\begin{document}

Конспектировала: Лиза Вахрамева.

\section{О следе матрицы}
Для квадратной матрицы \( A \in \RR^{n \times n} \) вводится понятие следа
матрицы:
\[ \tr(A) = \sum_{i=1}^n A_{ii}, \]
то есть след матрицы — это сумма её диагональных элементов. След матрицы обладает следующими основными свойствами, мотивирующими его введение и дальнейшее использование:
\begin{enumerate}
    \item \(\tr(AB) = \tr(BA)\),
    \item след матрицы равен сумме корней характеристического уравнения,
    \item след матрицы является скаляром: \( \tr(\cdot) \in \RR \).
\end{enumerate}
Покажем, что если у матрицы \(A\) есть \(n\) действительных собственных чисел
\(\lambda_1 \dots \lambda_n\), то \( \tr(A) = \sum_{i=1}^n \lambda_i \):
\[ A = PDP^{-1},\]
где \(P\) — матрица, составленная из собственных векторов \( v_1 \dots v_n \), соответсвующих собственным числам \( \lambda_1 \dots \lambda_n \), а матрица
\(D\) — диагональная из собственных чисел:
\[ \tr(A) = \tr(PDP^{-1}) = \tr(D P^{-1} P)
= \tr(D) = \sum_{i=1}^n \lambda_i. \]

\textbf{Упражнение:} дана регрессия \( \hat{y} = X\hat{\beta} \), построенная методом МНК по \(n\) наблюдениям, \(k\)  регрессорам. Вектор наблюдений \(y\) можно спроецировать на пространство наблюдений \(\RR^n\), в в котором строится предсказание, с помощью матрицы-шляпницы \(H\):
\[\hat{y} = Hy.\]
Нужно найти \( \tr(H) \). Рассмотрим два возможных решения:
\begin{enumerate}
    \item Содержательное. \\
    Известно, что след \(H\) равен сумме собственных значений \(H\). Также известно, что  \(H\) проецирует вектора на линейную оболочку векторов-столбцов матрицы \(X\), обозначим их \( c_1 \dots c_k \in \RR^n \). Так как всеобъемлющее пространство имеет размерность \(n\), то набор \( c_1 \dots c_k \) векторов можно дополнить ортогональным набором \( p_1 \dots p_{n-k} \) до базиса. Все вектора \( c_1 \dots c_k \) переходят сами в себя при проекции, поэтому являются собственными с собственными числами 1. Все вектора \( p_1 \dots p_{n-k} \)
    являются ортогональными оболочке, на которую строится проекция, и при проекции переходят в ноль, поэтому они являются собственными векторами с собственными числами 0. Получилось \( \lambda_1 \dots \lambda_k = 1 \) и \( \lambda_{k+1} \dots \lambda_n = 0 \).

    \[ \tr(H) = \sum_i \lambda_i = k \]

    \item По определению.
    \[ \tr(H) = \tr (X(X^TX)^{-1}X^T) =
      \tr ((X^TX)^{-1} X^TX) = \tr (I_{k \times k}) = k. \]
\end{enumerate}

\section{Ковариация и ковариационная матрица}
Пусть \(y_1, y_2 \in \RR\) — скалярные случайные величины,
\(a, b \in \RR^n\) — вектора-столбцы чисел.


Дисперсия СВ:
\[ \Var(y_1) = \E(y_1^2) - \big( \E(y_1) \big)^2. \]

Выборочная дисперсия столбца чисел:
\[ \sVar(a) = \frac{\sum_{i}^n (a_i - \bar{a})^2}{n-1} \]
квадрат длины центрированного вектора, деленный на размерность подпространства, в котором лежит вектор.

Ковариация двух СВ:
\[ \Cov(y_1, y_2) = \E y_1 y_2 - \E y_1 \E y_2= \E \bigg( (y_1 - \E y_1) (y_2 - \E y_2) \bigg).\]

Выборочная ковариация двух столбцов чисел:
\[ \sCov(a, b) = \frac{\sum_i (a_{i} - \bar{a}) (b_{i} - \bar{b})}{n-1}\]
скалярное произведение центрированных векторов, деленное на размерность пространства, в котором они лежат.

Корреляция двух СВ:
\[ \Corr(y_1, y_2) = \frac{\Cov(y_1, y_2)}
{\sqrt{\Var(y_1)} \sqrt{\Var(y_2)} }. \]


%добавим графики
\tikzset{every picture/.style={line width=0.75pt}} %set default line width to 0.75pt    
\begin{center}
\begin{tikzpicture}[x=0.75pt,y=0.75pt,yscale=-1,xscale=1]
%uncomment if require: \path (0,427); %set diagram left start at 0, and has height of 427

%Straight Lines [id:da7046667171619407] 
\draw    (148.22,324.57) -- (266.34,159.63) ;
\draw [shift={(267.5,158)}, rotate = 485.61] [color={rgb, 255:red, 0; green, 0; blue, 0 }  ][line width=0.75]    (10.93,-3.29) .. controls (6.95,-1.4) and (3.31,-0.3) .. (0,0) .. controls (3.31,0.3) and (6.95,1.4) .. (10.93,3.29)   ;

%Straight Lines [id:da244021284098491] 
\draw  [dash pattern={on 0.84pt off 2.51pt}]  (267.5,158) -- (146.5,159) ;


%Straight Lines [id:da1929269067918925] 
\draw [color={rgb, 255:red, 0; green, 0; blue, 0 }  ,draw opacity=1 ]   (265.5,321.06) -- (148.22,324.57) ;

\draw [shift={(267.5,321)}, rotate = 178.29] [color={rgb, 255:red, 0; green, 0; blue, 0 }  ,draw opacity=1 ][line width=0.75]    (10.93,-3.29) .. controls (6.95,-1.4) and (3.31,-0.3) .. (0,0) .. controls (3.31,0.3) and (6.95,1.4) .. (10.93,3.29)   ;
%Straight Lines [id:da8194662510474527] 
\draw    (148.22,324.57) -- (147.51,99) ;
\draw [shift={(147.5,97)}, rotate = 449.82] [color={rgb, 255:red, 0; green, 0; blue, 0 }  ][line width=0.75]    (10.93,-3.29) .. controls (6.95,-1.4) and (3.31,-0.3) .. (0,0) .. controls (3.31,0.3) and (6.95,1.4) .. (10.93,3.29)   ;

%Straight Lines [id:da738015664982508] 
\draw  [dash pattern={on 0.84pt off 2.51pt}]  (267.5,158) -- (267.5,321) ;


%Shape: Ellipse [id:dp038430448292362196] 
\draw   (81,302.5) .. controls (81,255.28) and (163.04,217) .. (264.25,217) .. controls (365.46,217) and (447.5,255.28) .. (447.5,302.5) .. controls (447.5,349.72) and (365.46,388) .. (264.25,388) .. controls (163.04,388) and (81,349.72) .. (81,302.5) -- cycle ;

% Text Node
\draw (282.61,149.18) node [scale=1]  {$a\ $};
% Text Node
\draw (170.61,87.18) node [scale=0.9]  {$1^{\trianglelefteq } =\left(\begin{matrix}
1\\
1\\
..\\
..\\
1
\end{matrix}\right) \ $};
% Text Node
\draw (268.61,338.18) node [scale=1]  {$a\ -\ \overline{a} \cdot 1^{\trianglelefteq } \ $};
% Text Node
\draw (351,195) node   {$\mathbb{R}^{n}$};
% Text Node
\draw (127.61,165.18) node [scale=1]  {$\overline{a} \cdot 1^{\trianglelefteq } \ $};
% Text Node
\draw (365,270) node   {$Col^{\perp } 1^{\trianglelefteq }$};
% Text Node
\draw (389,291) node [scale=0.7]  {$\dim Col^{\perp } 1^{\trianglelefteq } =n-1$};


\end{tikzpicture}
\end{center}

Выборочная корреляция двух столбцов чисел:
\[ \sCorr(a, b) = \frac{\sCov(a, b)}
{\sqrt{\sVar(a)} \sqrt{\sVar(b)} } =
\cos (a - \bar{a}, b - \bar{b}) = \cos\alpha. \]

%добавим графики
\begin{center}
 \begin{tikzpicture}[x=0.75pt,y=0.75pt,yscale=-1,xscale=1]
%uncomment if require: \path (0,427); %set diagram left start at 0, and has height of 427

%Straight Lines [id:da7715012207873877] 
\draw    (148.22,324.57) -- (266.34,159.63) ;
\draw [shift={(267.5,158)}, rotate = 485.61] [color={rgb, 255:red, 0; green, 0; blue, 0 }  ][line width=0.75]    (10.93,-3.29) .. controls (6.95,-1.4) and (3.31,-0.3) .. (0,0) .. controls (3.31,0.3) and (6.95,1.4) .. (10.93,3.29)   ;

%Straight Lines [id:da3628821017661845] 
\draw [color={rgb, 255:red, 0; green, 0; blue, 0 }  ,draw opacity=1 ]   (265.5,321.06) -- (148.22,324.57) ;

\draw [shift={(267.5,321)}, rotate = 178.29] [color={rgb, 255:red, 0; green, 0; blue, 0 }  ,draw opacity=1 ][line width=0.75]    (10.93,-3.29) .. controls (6.95,-1.4) and (3.31,-0.3) .. (0,0) .. controls (3.31,0.3) and (6.95,1.4) .. (10.93,3.29)   ;
%Straight Lines [id:da30792962820259984] 
\draw    (148.22,324.57) -- (147.51,99) ;
\draw [shift={(147.5,97)}, rotate = 449.82] [color={rgb, 255:red, 0; green, 0; blue, 0 }  ][line width=0.75]    (10.93,-3.29) .. controls (6.95,-1.4) and (3.31,-0.3) .. (0,0) .. controls (3.31,0.3) and (6.95,1.4) .. (10.93,3.29)   ;

%Straight Lines [id:da9947601288924183] 
\draw  [dash pattern={on 0.84pt off 2.51pt}]  (267.5,158) -- (267.5,321) ;


%Shape: Ellipse [id:dp6156140101536083] 
\draw   (74,302.5) .. controls (74,255.28) and (156.04,217) .. (257.25,217) .. controls (358.46,217) and (440.5,255.28) .. (440.5,302.5) .. controls (440.5,349.72) and (358.46,388) .. (257.25,388) .. controls (156.04,388) and (74,349.72) .. (74,302.5) -- cycle ;
%Straight Lines [id:da39060742340628074] 
\draw    (148.22,324.57) -- (207.96,111.93) ;
\draw [shift={(208.5,110)}, rotate = 465.69] [color={rgb, 255:red, 0; green, 0; blue, 0 }  ][line width=0.75]    (10.93,-3.29) .. controls (6.95,-1.4) and (3.31,-0.3) .. (0,0) .. controls (3.31,0.3) and (6.95,1.4) .. (10.93,3.29)   ;

%Straight Lines [id:da007539771018415964] 
\draw  [dash pattern={on 0.84pt off 2.51pt}]  (208.5,110) -- (208.5,273) ;


%Straight Lines [id:da6877394213991674] 
\draw [color={rgb, 255:red, 0; green, 0; blue, 0 }  ,draw opacity=1 ]   (206.98,274.3) -- (148.22,324.57) ;

\draw [shift={(208.5,273)}, rotate = 139.45] [color={rgb, 255:red, 0; green, 0; blue, 0 }  ,draw opacity=1 ][line width=0.75]    (10.93,-3.29) .. controls (6.95,-1.4) and (3.31,-0.3) .. (0,0) .. controls (3.31,0.3) and (6.95,1.4) .. (10.93,3.29)   ;
%Shape: Arc [id:dp5154040953447191] 
\draw  [draw opacity=0][line width=2.25]  (177.99,322.77) .. controls (186.62,322.49) and (193.5,317.23) .. (193.5,310.78) .. controls (193.5,304.48) and (186.92,299.31) .. (178.56,298.82) -- (177.27,310.78) -- cycle ; \draw  [color={rgb, 255:red, 208; green, 2; blue, 27 }  ,draw opacity=1 ][line width=2.25]  (177.99,322.77) .. controls (186.62,322.49) and (193.5,317.23) .. (193.5,310.78) .. controls (193.5,304.48) and (186.92,299.31) .. (178.56,298.82) ;

% Text Node
\draw (282.61,149.18) node [scale=1]  {$a\ $};
% Text Node
\draw (136.61,97.18) node [scale=0.9]  {$1^{\trianglelefteq }$};
% Text Node
\draw (268.61,338.18) node [scale=1]  {$a\ -\ \overline{a} \cdot 1^{\trianglelefteq } \ $};
% Text Node
\draw (351,195) node   {$\mathbb{R}^{n}$};
% Text Node
\draw (369,336) node   {$Col^{\perp } 1^{\trianglelefteq }$};
% Text Node
\draw (214.61,96.18) node [scale=1]  {$b\ $};
% Text Node
\draw (217.61,258.18) node [scale=1]  {$b\ -\ \overline{b} \cdot 1^{\trianglelefteq } \ $};
% Text Node
\draw (205.03,303.57) node [color={rgb, 255:red, 208; green, 2; blue, 27 }  ,opacity=1 ]  {$\alpha $};


\end{tikzpicture}
\end{center}

\textbf{Упражнение:}
Записать выборочную ковариационную матрицу для \(X\) в предположении, что переменные уже центрированы. Запишем матрицу \(X\) по строкам и по столбцам:
\[
X = \begin{bmatrix}
    \vdots &  & \vdots & \\
    c_1 & \dots & c_k \\
    \vdots &  & \vdots & \\
    \end{bmatrix}
    =
    \begin{bmatrix}
    \dots & r_{1}^T & \dots \\
    & \vdots & \\
    \dots & r_{n}^T & \dots \\
    \end{bmatrix}.
\]
Поймем, как выглядит \( \sVar(X) \):
\[
\sVar(X) = \begin{bmatrix}
\sVar(c_1) & \sCov(c_1, c_2) & \dots & \sCov(c_1, c_k) \\
\sCov(c_2, c_1) & \sVar(c_2) & \dots & \sCov(c_2, c_k) \\
\vdots & & \ddots & \vdots\\
\sCov(c_k, c_1) & \sCov(c_k, c_2) & \dots & \sVar(c_k, c_k).
\end{bmatrix}
\]
С учетом того что \( c_1 \dots c_k \) центрированы, ковариационную матрицу можно найти следующими способами:
\[ \sVar(X) = \frac{X^TX}{n-1} = \frac{\sum_i^n r_i r_i^T}{n-1}.\]
Чтобы удостовериться в правильности ответов, рассмотрим одну из ячеек ковариационной матрицы:
\[\sVar(X)_{21} = \frac{c_2^Tc_1}{n-1} = \frac{\sum_i^n r_{i2} r_{i1} }{n-1}. \]

\section{Метод главных компонент}
В методе главных компонент рассматривается способ понижения размерности при минимальных потерях в разбросе данных.
%\textbf{Постановка задачи: снизить размерность пространства.}
\[ \min_{\mu, V: V^TV=I, \lambda_i} \sum_{i}^n \left\|r_i - \mu - V\lambda_i \right\|, \]
\(r_i \in \RR^k\) — \(i\)-ая строка матрицы \(X\), записанная в столбец,
\(\mu \in \RR^k\) — вектор средних, \(V \in \RR^{k \times p}\)  —  матрица, столбцы которой ортонормальны: \( V^TV=I \), \( \lambda_i \in \RR^p\), \(k\)  —  исходная размерность, \(p\)  —  желаемая.

Предположим, что \(\mu\) и \(V\) найдены, выразим \(\lambda_i\). Заметим, что для фиксированного \(i\) задача превращается в аналогичную задаче поиска коэффициентов линейной регрессии , поэтому можно сразу выписать ответ:
\[\lambda_i = (V^TV)^{-1}V^T(r_i-\mu) = V^T(r_i-\mu).\]
Далее будем считать, что переменные заранее центрированы, поэтому положим \(\mu=0\).
Подставим выражение для \(\lambda_i\) в исходную задачу:
\[Q = \left\| r_i - VV^T r_i \right\| \to \min_{V: V^TV=I} .\]
Распишем:
\[Q = \sum_i^n (r_i - VV^Tr_i)^T (r_i - VV^Tr_i) =
\sum_i r_i^T (I-VV^T)(I-VV^T)r_i.\]
Заметим, что \(VV^T\)  —  это матрица-шляпница, проецирующая \(r_i\) на линейную оболочку из \(v_1 \dots v_p\) столбцов матрицы \(V\):
\[ H = V (V^TV)^{-1} V^T = VV^T.\]
Тогда матрица \( I-VV^T \) является матрицей, проецирующей вектора \(r_i\) на ортогональное дополнение к пространству, натянутому на оболочку \(v_1 \dots v_p\).
Проекция на подространство \(L\) вектора \(x \in L\) уже лежащего в этом подпространстве, равна \(x\), поэтому  \( (I-VV^T)(I-VV^T)x=(I-VV^T)x \quad \forall x\).
Получили:
\[ Q = \sum_i^n r_i^T(I - VV^T)r_i. \]
Раскроем скобки:
\[Q = \sum_i^n r_i^T r_i - r_i^T VV^T r_i \to \min_{V: V^TV=I}.\]
Перейдем к эквивалентной задаче:
\[Q=r_i^T VV^T r_i \to \max_{V: V^TV=I}.\]
Tак как \(Q \in \RR\) - скаляр, то можно записать:
\[Q = \tr (\sum_i^n r_i^T VV^T r_i ) =
 \tr (VV^T \sum_i^n r_i r_i^T) \]
 \[
 = \tr (VV^T (n-1)\sVar(X))=
 (n-1)\tr (V \sVar(X) V^T )
 = (n-1) \sum_{j}^p v_j^T \sVar(X) v_j \]

Перейдем к эквивалентной задаче:
\[Q' = \sum_{j}^p v_j^T S v_j \to \max,\]
где \(S = \sVar(X)\). \(Q'\) интерпретируется как выборочная дисперсия линейной комбинации столбцов матрицы \(X\), взятых с весами вектора \(v_j\).

\textbf{Упражнение}. Рассмотрим матрицу \(X \in \RR^{n \times 2}\) и ее ковариационную матрицу:
\[
X = \begin{bmatrix}
    \vdots & \vdots & \\
    c_1 &  c_k \\
    \vdots &   \vdots & \\
    \end{bmatrix},
\]

\[
\sVar(X) =
\begin{bmatrix}
5 & -1 \\
-1 & 16
\end{bmatrix}.
\]


Будем искать выборочную дисперсию \(z = 3c_1 + 6c_2\).
\[ \Var(3c_1+6c_2)=9\Var(c_1) + 36 \Var(c_2) + 18 \Cov(c_1, c_2) = 9 \cdot 5 + 36 \cdot 16 + 18 \cdot (-1) = 603.\]
\[
\begin{bmatrix}
3 & -6 \\
\end{bmatrix}
\begin{bmatrix}
5 & -1 \\
-1 & 16
\end{bmatrix}.
\begin{bmatrix}
3\\
6
\end{bmatrix}=9 \cdot 5 + 36 \cdot 16 + 18 \cdot (-1) = 603.
\]
Получилось, что выборочную дисперсию можно искать по формуле для дисперсии суммы, а можно через построение квадратичной формы.

Можно снова переписать оптимизируемый функционал:
\[
V = \begin{bmatrix}
    \vdots & & \vdots & \\
    v_1 & \dots &v_p \\
    \vdots & & \vdots & \\
    \end{bmatrix},
\]

\[ \sum_{j}^p \sVar(Xv_j) \to \max_{V: V^TV=I}. \]

\textbf{Упражнение:} Найти главную компоненту \(X\).
\[
X =
\begin{bmatrix}
3 & 5\\
2 & 1\\
1 & 3\\
\end{bmatrix}
\]
Сначала нужно центрировать и нормировать данные:
\[
\begin{bmatrix}
3 & 5\\
2 & 1\\
1 & 3\\
\end{bmatrix}
\to
\begin{bmatrix}
1 & 2\\
0 & -2\\
-1 & 0\\
\end{bmatrix}
\to
\begin{bmatrix}
1 & 1\\
0 & -1\\
-1 & 0\\
\end{bmatrix}
=X'
\]
С помощью метода главных компонент будем искать вектор \(v_1 \in \RR^2\) —  веса, с которыми нужно взять линейную комбинацию столбцов матрицы \(X\), чтобы получить первую главную компоненту. Введем обозначение:
\[v_1
= \begin{bmatrix}
\alpha\\
\beta
\end{bmatrix},
\]
тогда требования на ортонормированность базиса из векторов \(v\) в случае с одной главной компонентой  — только нормированность) можно записать так:
\[ \left\| v_1 \right\| = 1 \to \alpha^2 + \beta^2 = 1. \]
Запишем оптимизируемый функционал:
\[
Q = \sVar \bigg(
\alpha \begin{bmatrix}
1\\
0\\
-1\\
\end{bmatrix}
+
\beta \begin{bmatrix}
1\\
-1\\
0\\
\end{bmatrix}
\bigg)
\to \max_{\alpha, \beta : \alpha^2+\beta^2=1}
\]
Чтобы посчитать значение этого функционала, нужно посчитать ковариационную матрицу \(X'\):
\[
\sVar(X')=
\begin{bmatrix}
1&\frac{1}{2}\\
\frac{1}{2}&1
\end{bmatrix}.
\]
Тогда
\[Q = \alpha^2 + \beta^2 + 2\alpha \beta \frac{1}{2} = 1+\alpha \beta.\]
Решением задачи:
\[
\begin{cases}
\alpha \beta \to \max \\
\alpha^2 + \beta^2 = 1
\end{cases}
\]

являются точки \(\alpha=\beta=\frac{1}{\sqrt{2}}\) и \(\alpha=\beta=-\frac{1}{\sqrt{2}}\). Знак здесь задает только ориентацию вектора, поэтому можно рассматривать любую из точек.

Теперь можно записать главную компоненту:
\[pc_1 =
\frac{1}{\sqrt{2}} \begin{bmatrix}
1\\
0\\
-1\\
\end{bmatrix}
+
\frac{1}{\sqrt{2}} \begin{bmatrix}
1\\
-1\\
0\\
\end{bmatrix}
=
\begin{bmatrix}
\frac{2}{\sqrt{2}}\\~\\
\frac{-1}{\sqrt{2}}\\~\\
\frac{-1}{\sqrt{2}}\\
\end{bmatrix}
\]
Поиск нескольких главных компонент можно осуществлять, последовательно решая задачи:
\[
\begin{cases}
\sVar(Xv_1) \to \max \\
\left\| v_1 \right\| = 1
\end{cases}
\]

\[
\begin{cases}
\sVar(Xv_2) \to \max \\
\left\| v_2 \right\| = 1 \\
v_2^Tv_1 = 0
\end{cases}
\]

\[
\begin{cases}
\sVar(Xv_3) \to \max \\
\left\| v_3 \right\| = 1 \\
v_3^Tv_1 = 0\\
v_2^Tv_1 = 0
\end{cases}
\]

\textbf{Упражнение:} Если \(V^TV=I\), то \(V\) сохраняет длины.
\[ \left\| Va \right\| = a^TV^TVa = \left\| a \right\|. \]

\textbf{Упражнение:} Если \( \left\| a \right\| = 1 \),
то
\[ \max_a a^T
\begin{bmatrix}
1 & 0 & 0 \\
0 & 3 & 0 \\
0 & 0 & 1 \\
\end{bmatrix}
a =
\begin{bmatrix}
0\\
1\\
0
\end{bmatrix}.
\]

\section{Связь с SVD-разложением}
Рассмотрим поиск первой главной компоненты:
\[ \max_{v_1: \left\| v_1 \right\| = 1} \sVar(Xv_1)
= \max_{v_1: \left\| v_1 \right\| = 1} v_1^T \frac{X^TX}{n-1} v_1 =
\max_{v_1: \left\| v_1 \right\| = 1} v_1^T X^TX v_1 \]

SVD-разложение:
\[ X = U\Sigma V^T \]
\[ X^TX = V \Sigma^T \Sigma V^T \]
Подставим в оптимизируемый функционал:
\[ \max_{v_1: \left\| v_1 \right\| = 1} v_1^T V \Sigma^T \Sigma V v_1.\]
Так как \(V^TV=I\), то \(V\) сохраняет длины, поэтому
\( \left\| Vv_1 \right\| = \left\| v_1^T V \right\| = 1 \iff \left\| v_1 \right\|=1\), а значит можно переписать задачу:
\[ \max_{v_1': \left\| v_1' \right\| = 1} v_1'^T \Sigma^T \Sigma V v_1'
=
\max_{v_1': \left\| v_1' \right\| = 1}
v_1'
\begin{bmatrix}
\lambda_1 & \dots & 0\\
& \ddots &\\
0 & \dots & \lambda_n
\end{bmatrix} v_1' =
\begin{bmatrix}
1\\
0\\
\vdots\\
0
\end{bmatrix}\]
здесь \(\lambda_1 \dots \lambda_n\)  —  собственные числа матрицы \(\Sigma^T\Sigma\),
отсортированные по убыванию.

Сейчас мы нашли \(v_1'\), но нужно найти \(v_1\).
\[V^Tv_1 =
\begin{bmatrix}
1\\
0\\
\vdots\\
0
\end{bmatrix}\]

\[VV^Tv_1 = v_1 = V
\begin{bmatrix}
1\\
0\\
\vdots\\
0
\end{bmatrix}\]
Получилось, что искомый столбец \(v_1\) для PCA  —  это столбец матрицы \(V\) из SVD разложения \(X\), соответсвующий первому по величине значению в матрице \( \Sigma\) (\(\Sigma_{ii} = \sqrt{\lambda_i}\)).

\[
pc_1 = Xv_1 = (U\Sigma V^T)v_1 =
U \Sigma
\begin{bmatrix}
1\\
0\\
\vdots\\
0
\end{bmatrix}=
\sqrt{\lambda_1}u_1.
\]
Это верно для всех рассматриваемых главных компонент:
\[pc_j = \sqrt{\lambda_j} u_j.\]

\end{document}