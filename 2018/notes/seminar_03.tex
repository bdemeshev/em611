\documentclass[12pt]{article} % размер шрифта

\usepackage{tikz} % картинки в tikz
\usepackage{microtype} % свешивание пунктуации

\usepackage{array} % для столбцов фиксированной ширины

\usepackage{url} % для вставки ссылок \url{...}

\usepackage{indentfirst} % отступ в первом параграфе

\usepackage{sectsty} % для центрирования названий частей
\allsectionsfont{\centering} % приказываем центрировать все sections

\usepackage{amsthm} % теоремы и доказательства

\theoremstyle{definition} % прямой шрифт в условии теорем
\newtheorem{theorem}{Теорема}[section]


\usepackage{amsmath, amssymb} % куча стандартных математических плюшек

\usepackage[top=2cm, left=1.5cm, right=1.5cm, bottom=2cm]{geometry} % размер текста на странице

\usepackage{lastpage} % чтобы узнать номер последней страницы

\usepackage{enumitem} % дополнительные плюшки для списков
%  например \begin{enumerate}[resume] позволяет продолжить нумерацию в новом списке
\usepackage{caption} % подписи к картинкам без плавающего окружения figure


\usepackage{fancyhdr} % весёлые колонтитулы
\pagestyle{fancy}
\lhead{Эконометрика, финтех}
\chead{}
\rhead{2018-10-06, встреча 3}
\lfoot{}
\cfoot{}
\rfoot{\thepage/\pageref{LastPage}}
\renewcommand{\headrulewidth}{0.4pt}
\renewcommand{\footrulewidth}{0.4pt}



\usepackage{todonotes} % для вставки в документ заметок о том, что осталось сделать
% \todo{Здесь надо коэффициенты исправить}
% \missingfigure{Здесь будет картина Последний день Помпеи}
% команда \listoftodos — печатает все поставленные \todo'шки

\usepackage{booktabs} % красивые таблицы
% заповеди из документации:
% 1. Не используйте вертикальные линии
% 2. Не используйте двойные линии
% 3. Единицы измерения помещайте в шапку таблицы
% 4. Не сокращайте .1 вместо 0.1
% 5. Повторяющееся значение повторяйте, а не говорите "то же"

\usepackage{fontspec} % поддержка разных шрифтов
\usepackage{polyglossia} % поддержка разных языков

\setmainlanguage{russian}
\setotherlanguages{english}

\setmainfont{Linux Libertine O} % выбираем шрифт
% если Linux Libertine не установлен, то
% можно также попробовать Helvetica, Arial, Cambria и т.Д.

% чтобы использовать шрифт Linux Libertine на личном компе,
% его надо предварительно скачать по ссылке
% http://www.linuxlibertine.org/index.php?id=91&L=1

% на сервисах типа sharelatex.com этот шрифт есть :)

\newfontfamily{\cyrillicfonttt}{Linux Libertine O}
% пояснение зачем нужно шаманство с \newfontfamily
% http://tex.stackexchange.com/questions/91507/

\AddEnumerateCounter{\asbuk}{\russian@alph}{щ} % для списков с русскими буквами
\setlist[enumerate, 2]{label=\asbuk\cdot),ref=\asbuk\cdot} % списки уровня 2 будут буквами а) б) ...

%% эконометрические и вероятностные сокращения
\DeclareMathOperator{\Cov}{Cov}
\DeclareMathOperator{\Corr}{Corr}
\DeclareMathOperator{\Var}{Var}
\DeclareMathOperator{\E}{E}
\def \hb{\hat{\beta}}
\def \hs{\hat{\sigma}}
\def \htheta{\hat{\theta}}
\def \s{\sigma}
\def \hy{\hat{y}}
\def \hY{\hat{Y}}
\def \v1{\vec{1}}
\def \e{\varepsilon}
\def \he{\hat{\e}}
\def \z{z}
\def \hVar{\widehat{\Var}}
\def \hCorr{\widehat{\Corr}}
\def \hCov{\widehat{\Cov}}
\def \cN{\mathcal{N}}


\begin{document}

Конспектировала: Лиза Вахрамева.

\section{О следе матрицы}
Для квадратной матрицы $A \in \mathbb{R}^{n \times n}$ вводится понятие следа
матрицы:
$$\text{tr}(A) = \sum_{i=1}^n A_{ii},$$
то есть след матрицы — это сумма её диагональных элементов. След матрицы обладает следующими основными свойствами, мотивирующими его введение и дальнейшее использование:
\begin{enumerate}
    \item $\text{tr}(AB) = \text{tr}(BA)$,
    \item след матрицы равен сумме корней характеристического уравнения,\\
    \item след матрицы является скаляром: $\text{tr}(\cdot) \in \mathbb{R}$.
\end{enumerate}
Покажем, что если у матрицы $A$ есть $n$ действительных собственных чисел
$\lambda_1 \dots \lambda_n$, то $\text{tr}(A) = \sum_{i=1}^n \lambda_i$:
$$A = PDP^{-1},$$
где $P$ — матрица, составленная из собственных векторов $v_1 \dots v_n$, соответсвующих собственным числам $\lambda_1 \dots \lambda_n$, а матрица
$D$ — диагональная из собственных чисел:
$$\text{tr}(A) = \text{tr}(PDP^{-1}) = \text{tr}(D PP^{-1})
= \text{tr}(D) = \sum_{i=1}^n \lambda_i.$$

\textbf{Упражнение:} дана регрессия $\hat{y} = X\hat{\beta}$, построенная методом МНК по $n$ - наблюдениям, $k$ - регрессорам. Вектор наблюдений $y$ можно спроецировать на пространство наблюдений $\mathbb{R}^n$, в в котором строится предсказание, с помощью матрицы-шляпницы $H$:
$$\hat{y} = Hy.$$
Нужно найти $\text{tr}(H)$. Рассмотрим два возможных решения:
\begin{enumerate}
    \item Содержательное. \\
    Известно, что след $H$ равен сумме собственных значений $H$. Также известно, что  $H$ проецирует вектора на линейную оболочку векторов-столбцов матрицы $X$, обозначим их $c_1 \dots c_k \in \mathbb{R}^n$. Так как всеобъемлющее пространство имеет размерность $n$, то набор $c_1 \dots c_k$ векторов можно дополнить ортогональным набором $p_1 \dots p_{n-k}$ до базиса. Все вектора $c_1 \dots c_k$ переходят сами в себя при проекции, поэтому являются собственными с собственными числами 1. Все вектора $p_1 \dots p_{n-k}$
    являются ортогональными оболочке, на которую строится проекция, и при проекции переходят в ноль, поэтому они являются собственными векторами с собственными числами 0. Получилось $\lambda_1 \dots \lambda_k = 1$ и $\lambda_{k+1} \dots \lambda_n = 0$.

    $$\text{tr}(H) = \sum_i \lambda_i = k$$

    \item По определению.
    $$ \text{tr}(H) = \text{tr} (X(X^TX)^{-1}X^T) =
      \text{tr} ((X^TX)^{-1} X^TX) = \text{tr} (I_{k \times k}) = k.$$
\end{enumerate}

\section{Ковариация и ковариационная матрица}
Пусть $y_1, y_2 \in \mathbb{R}$ — скалярные случайные величины,
$a, b \in \mathbb{R}^n$ — вектора-столбцы чисел.


Дисперсия СВ:
$$\text{Var}(y_1) = \mathbb{E}(y_1^2) - \big( \mathbb{E}(y_1) \big)^2.$$

Выборочная дисперсия столбца чисел:
$$\text{sVar}(a) = \frac{\sum_{i}^n (a_i - \bar{a})^2}{n-1}$$
квадрат длины центрированного вектора, деленный на размерность подпространства, в котором лежит вектор.

Ковариация двух СВ:
$$\text{Cov}(y_1, y_2) = \mathbb{E}y_1 y_2 - \mathbb{E}y_1 \mathbb{E}y_2= \mathbb{E} \bigg( (y_1 - \mathbb{E}y_1) (y_2 - \mathbb{E}y_2) \bigg).$$

Выборочная ковариация двух столбцов чисел:
$$\text{sCov}(a, b) = \frac{\sum_i (a_{i} - \bar{a}) (b_{i} - \bar{b})}{n-1}$$
скалярное произведение центрированных векторов, деленное на размерность пространства, в котором они лежат.

Корреляция двух СВ:
$$\text{Corr}(y_1, y_2) = \frac{\text{Cov}(y_1, y_2)}
{\sqrt{\text{Var}(y_1)} \sqrt{\text{Var}(y_2)} }. $$

Выборочная корреляция двух столбцов чисел:
$$\text{sCorr}(a, b) = \frac{\text{sCov}(a, b)}
{\sqrt{\text{sVar}(a)} \sqrt{\text{sVar}(b)} } =
\text{cos} (a - \bar{a}, b - \bar{b}). $$

\textbf{Упражнение:}
Записать выборочную ковариационную матрицу для $X$ в предположении, что переменные уже центрированы. Запишем матрицу $X$ по строкам и по столбцам:
$$
X = \begin{bmatrix}
    \vdots &  & \vdots & \\
    c_1 & \dots & c_k \\
    \vdots &  & \vdots & \\
    \end{bmatrix}
    =
    \begin{bmatrix}
    \dots & r_{1}^T & \dots \\
    & \vdots & \\
    \dots & r_{n}^T & \dots \\
    \end{bmatrix}.
$$
Поймем, как выглядит $\text{sVar}(X)$:
$$
\text{sVar}(X) = \begin{bmatrix}
\text{sVar}(c_1) & \text{sCov}(c_1, c_2) & \dots & \text{sCov}(c_1, c_k) \\
\text{sCov}(c_2, c_1) & \text{sVar}(c_2) & \dots & \text{sCov}(c_2, c_k) \\
\vdots & & \ddots & \vdots\\
\text{sCov}(c_k, c_1) & \text{sCov}(c_k, c_2) & \dots & \text{sVar}(c_k, c_k).
\end{bmatrix}
$$
С учетом того что $c_1 \dots c_k$ центрированы, ковариационную матрицу можно найти следующими способами:
$$\text{sVar}(X) = \frac{X^TX}{n-1} = \frac{\sum_i^n r_i r_i^T}{n-1}.$$
Чтобы удостовериться в правильности ответов, рассмотрим одну из ячеек ковариационной матрицы:
$$\text{sVar}(X)_{21} = \frac{c_2^Tc_1}{n-1} = \frac{\sum_i^n r_{i2} r_{i1} }{n-1}. $$

\section{Метод главных компонент}
В методе главных компонент рассматривается способ понижения размерности при минимальной потерях в разбросе данных.
%\textbf{Постановка задачи: снизить размерность пространства.}
$$\min_{\mu, V: V^TV=I, \lambda_i} \sum_{i}^n \left\|r_i - \mu - V\lambda_i \right\|, $$
$r_i \in \mathbb{R}^k$ — $i$-ая строка матрицы $X$, записанная в столбец,
$\mu \in \mathbb{R}^k$ — вектор средних, $V \in \mathbb{R}^{k \times p}$  —  матрица, столбцы которой ортонормальны: $V^TV=I$, $\lambda_i \in \mathbb{R}^p$, $k$  —  исходная размерность, $p$  —  желаемая.

Предположим, что $\mu$ и $V$ найдены, выразим $\lambda_i$. Заметим, что для фиксированного $i$ задача превращается в аналогичную задаче поиска коэффициентов линейной регрессии , поэтому можно сразу выписать ответ:
$$\lambda_i = (V^TV)^{-1}V^T(r_i-\mu) = V^T(r_i-\mu).$$
Далее будем считать, что переменные заранее центрированы, поэтому положим $\mu=0$.
Подставим выражение для $\lambda_i$ в исходную задачу:
$$Q = \left\| r_i - VV^T r_i \right\| \to \min_{V: V^TV=I} .$$
Распишем:
$$Q = \sum_i^n (r_i - VV^Tr_i)^T (r_i - VV^Tr_i) =
\sum_i r_i^T (I-VV^T)(I-VV^T)r_i.$$
Заметим, что $VV^T$  —  это матрица-шляпница, проецирующая $r_i$ на линейную оболочку из $v_1 \dots v_p$ столбцов матрицы $V$:
$$H = V (V^TV)^{-1} V^T = VV^T.$$
Тогда матрица $I-VV^T$ является матрицей, проецирующей вектора $r_i$ на ортогональное дополнение к пространству, натянутому на оболочку $v_1 \dots v_p$.
Проекция на подространство $L$ вектора $x \in L$ уже лежащего в этом подпространстве, равна $x$, поэтому  $(I-VV^T)(I-VV^T)x=(I-VV^T)x \quad \forall x$.
Получили:
$$Q = \sum_i^n r_i^T(I - VV^T)r_i.$$
Раскроем скобки:
$$Q = \sum_i^n r_i^T r_i - r_i^T VV^T r_i \to \min_{V: V^TV=I}.$$
Перейдем к эквивалентной задаче:
$$Q=r_i^T VV^T r_i \to \max_{V: V^TV=I}.$$
Tак как $Q \in \mathbb{R}$ - скаляр, то можно записать:
$$Q = \text{tr} (\sum_i^n r_i^T VV^T r_i ) =
 \text{tr} (VV^T \sum_i^n r_i^T r_i) $$
 $$
 = \text{tr} (VV^T (n-1)\text{sVar}(X))=
 (n-1)\text{tr} (V \text{sVar}(X) V^T )
 = (n-1) V \text{sVar}(X) V^T$$

Перейдем к эквивалентной задаче:
$$Q' = \sum_{j}^p v_j^T S v_j \to \max,$$
где $S = \text{sVar}(X)$. $Q'$ интерпретируется как выборочная дисперсия линейной комбинации столбцов матрицы $X$, взятых с весами вектора $v_j$.

\textbf{Упражнение}. Рассмотрим матрицу $X \in \mathbb{R}^{n \times 2}$ и ее ковариационную матрицу:
$$
X = \begin{bmatrix}
    \vdots & \vdots & \\
    c_1 &  c_k \\
    \vdots &   \vdots & \\
    \end{bmatrix},
$$
$$
\text{sVar}(X) =
\begin{bmatrix}
5 & -1 \\
-1 & 16
\end{bmatrix}.
$$
Будем искать выборочную дисперсию $z = 3c_1 + 6c_2$.
$$\text{Var}(3c_1+6c_2)=9\text{Var}(c_1) + 36 \text{Var}(c_2) + 18 \text{Cov}(c_1, c_2) = 9 \cdot 5 + 36 \cdot 16 + 18 \cdot (-1) = 603.$$
$$
\begin{bmatrix}
3 & -6 \\
\end{bmatrix}
\begin{bmatrix}
5 & -1 \\
-1 & 16
\end{bmatrix}.
\begin{bmatrix}
3\\
6
\end{bmatrix}=9 \cdot 5 + 36 \cdot 16 + 18 \cdot (-1) = 603.
$$
Получилось, что выборочную дисперсию можно искать по формуле для дисперсии суммы, а можно через построение квадратичной формы.

Можно снова переписать оптимизируемый функционал:
$$
V = \begin{bmatrix}
    \vdots & & \vdots & \\
    v_1 & \dots &v_p \\
    \vdots & & \vdots & \\
    \end{bmatrix},
$$
$$\sum_{j}^p \text{sVar}(Xv_j) \to \max_{V: V^TV=I}.$$

\textbf{Упражнение:} Найти главную компоненту $X$.
$$
X =
\begin{bmatrix}
3 & 5\\
2 & 1\\
1 & 3\\
\end{bmatrix}
$$
Сначала нужно центрировать и нормировать данные:
$$
\begin{bmatrix}
3 & 5\\
2 & 1\\
1 & 3\\
\end{bmatrix}
\to
\begin{bmatrix}
1 & 2\\
0 & -2\\
-1 & 0\\
\end{bmatrix}
\to
\begin{bmatrix}
1 & 1\\
0 & -1\\
-1 & 0\\
\end{bmatrix}
=X'
$$
С помощью метода главных компонент будем искать вектор $v_1 \in \mathbb{R}^2$  —  веса, с которыми нужно взять линейную комбинацию столбцов матрицы $X$, чтобы получить первую главную компоненту. Введем обозначение:
$$v_1
= \begin{bmatrix}
\alpha\\
\beta
\end{bmatrix},
$$
тогда требования на ортонормированность базиса из векторов $v$ (в случае с одной главной компонентой  — только нормированность) можно записать так:
$$\left\| v_1 \right\| = 1 \to \alpha^2 + \beta^2 = 1.$$
Запишем оптимизируемый функционал:
$$
Q = \text{sVar} \bigg(
\alpha \begin{bmatrix}
1\\
0\\
-1\\
\end{bmatrix}
+
\beta \begin{bmatrix}
1\\
-1\\
0\\
\end{bmatrix}
\bigg)
\to \max_{\alpha, \beta : \alpha^2+\beta^2=1}
$$
Чтобы посчитать значение этого функционала, нужно посчитать ковариационную матрицу $X'$:
$$
\text{sVar}(X')=
\begin{bmatrix}
1&\frac{1}{2}\\
\frac{1}{2}&1
\end{bmatrix}.
$$
Тогда
$$Q = \alpha^2 + \beta^2 + 2\alpha \beta \frac{1}{2} = 1+\alpha \beta.$$
Решением задачи:
$$
\begin{cases}
\alpha \beta \to \max \\
\alpha^2 + \beta^2 = 1
\end{cases}
$$
являются точки $\alpha=\beta=\frac{1}{\sqrt{2}}$ и $\alpha=\beta=-\frac{1}{\sqrt{2}}$. Знак здесь задает только ориентацию вектора, поэтому можно рассматривать любую из точек.

Теперь можно записать главную компоненту:
$$pc_1 =
\frac{1}{\sqrt{2}} \begin{bmatrix}
1\\
0\\
-1\\
\end{bmatrix}
+
\frac{1}{\sqrt{2}} \begin{bmatrix}
1\\
-1\\
0\\
\end{bmatrix}
=
\begin{bmatrix}
\frac{2}{\sqrt{2}}\\~\\
\frac{-1}{\sqrt{2}}\\~\\
\frac{-1}{\sqrt{2}}\\
\end{bmatrix}
$$
Поиск нескольких главных компонент можно осуществлять, последовательно решая задачи:
$$
\begin{cases}
\text{sVar}(Xv_1) \to \max \\
\left\| v_1 \right\| = 1
\end{cases}
$$

$$
\begin{cases}
\text{sVar}(Xv_2) \to \max \\
\left\| v_2 \right\| = 1 \\
v_2^Tv_1 = 0
\end{cases}
$$

$$
\begin{cases}
\text{sVar}(Xv_3) \to \max \\
\left\| v_3 \right\| = 1 \\
v_3^Tv_1 = 0\\
v_2^Tv_1 = 0
\end{cases}
$$

\textbf{Упражнение:} Если $V^TV=I$, то $V$ сохраняет длины.
$$\left\| Va \right\| = a^TV^TVa = \left\| a \right\|.$$

\textbf{Упражнение:} Если $\left\| a \right\| = 1$,
то
$$\max_a a^T
\begin{bmatrix}
1 & 0 & 0 \\
0 & 3 & 0 \\
0 & 0 & 1 \\
\end{bmatrix}
a =
\begin{bmatrix}
0\\
1\\
0
\end{bmatrix}.
$$
\section{Связь с SVD-разложением}
Рассмотрим поиск первой главной компоненты:
$$\max_{v_1: \left\| v_1 \right\| = 1} \text{sVar}(Xv_1)
= \max_{v_1: \left\| v_1 \right\| = 1} v_1^T \frac{X^TX}{n-1} v_1 =
\max_{v_1: \left\| v_1 \right\| = 1} v_1^T X^TX v_1 $$

SVD-разложение:
$$X = U\Sigma V^T$$
$$X^TX = V \Sigma^T \Sigma V^T$$
Подставим в оптимизируемый функционал:
$$ \max_{v_1: \left\| v_1 \right\| = 1} v_1^T V \Sigma^T \Sigma V v_1.$$
Так как $V^TV=I$, то $V$ сохраняет длины, поэтому
$\left\| Vv_1 \right\| = \left\| v_1^T V \right\| = 1 \iff \left\| v_1 \right\|=1$, а значит можно переписать задачу:
$$ \max_{v_1': \left\| v_1' \right\| = 1} v_1'^T \Sigma^T \Sigma V v_1'
=
\max_{v_1': \left\| v_1' \right\| = 1}
v_1'
\begin{bmatrix}
\lambda_1 & \dots & 0\\
& \ddots &\\
0 & \dots & \lambda_n
\end{bmatrix} v_1' =
\begin{bmatrix}
1\\
0\\
\vdots\\
0
\end{bmatrix}$$
здесь $\lambda_1 \dots \lambda_n$  —  собственные числа матрицы $\Sigma^T\Sigma$,
отсортированные по убыванию.

Сейчас мы нашли $v_1'$, но нужно найти $v_1$.
$$V^Tv_1 =
\begin{bmatrix}
1\\
0\\
\vdots\\
0
\end{bmatrix}$$
$$VV^Tv_1 = v_1 = V
\begin{bmatrix}
1\\
0\\
\vdots\\
0
\end{bmatrix}$$
Получилось, что искомый столбец $v_1$ для PCA  —  это столбец матрицы $V$ из SVD разложения $X$,
соответсвующий первому по величине значению в матрице $\Sigma$ ($\Sigma_{ii} = \sqrt{\lambda_i}$).

$$pc_1 = Xv_1 = (U\Sigma V^T)v_1 =
U \Sigma
\begin{bmatrix}
1\\
0\\
\vdots\\
0
\end{bmatrix}=
\sqrt{\lambda_1}u_1.
$$
Это верно для всех рассматриваемых главных компонент:
$$pc_j = \sqrt{\lambda_j} u_j.$$

\end{document}
