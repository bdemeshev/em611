\documentclass[12pt]{article} % размер шрифта
\usepackage{varwidth}
\usepackage{tikz} % картинки в tikz
\usepackage{microtype} % свешивание пунктуации

\usepackage{array} % для столбцов фиксированной ширины

\usepackage{url} % для вставки ссылок \url{...}

\usepackage{indentfirst} % отступ в первом параграфе

\usepackage{sectsty} % для центрирования названий частей
\allsectionsfont{\centering} % приказываем центрировать все sections

\usepackage{amsthm} % теоремы и доказательства

\theoremstyle{definition} % прямой шрифт в условии теорем
\newtheorem{theorem}{Теорема}[section]


\usepackage{amsmath, amssymb} % куча стандартных математических плюшек

\usepackage[top=2cm, left=1.5cm, right=1.5cm, bottom=2cm]{geometry} % размер текста на странице

\usepackage{lastpage} % чтобы узнать номер последней страницы

\usepackage{enumitem} % дополнительные плюшки для списков
%  например \begin{enumerate}[resume] позволяет продолжить нумерацию в новом списке
\usepackage{caption} % подписи к картинкам без плавающего окружения figure

\usepackage{hyperref} % гиперссылки

\usepackage{verbatim} % побуквенный вывод

\usepackage{fancyhdr} % весёлые колонтитулы
\pagestyle{fancy}
\lhead{Эконометрика, финтех}
\chead{}
\rhead{2018-12-18, встреча 12}
\lfoot{}
\cfoot{}
\rfoot{\thepage/\pageref{LastPage}}
\renewcommand{\headrulewidth}{0.4pt}
\renewcommand{\footrulewidth}{0.4pt}



\usepackage{todonotes} % для вставки в документ заметок о том, что осталось сделать
% \todo{Здесь надо коэффициенты исправить}
% \missingfigure{Здесь будет картина Последний день Помпеи}
% команда \listoftodos — печатает все поставленные \todo'шки

\usepackage{booktabs} % красивые таблицы
% заповеди из документации:
% 1. Не используйте вертикальные линии
% 2. Не используйте двойные линии
% 3. Единицы измерения помещайте в шапку таблицы
% 4. Не сокращайте .1 вместо 0.1
% 5. Повторяющееся значение повторяйте, а не говорите "то же"

\usepackage{fontspec} % поддержка разных шрифтов
\usepackage{polyglossia} % поддержка разных языков

\setmainlanguage{russian}
\setotherlanguages{english}

\setmainfont{Linux Libertine O} % выбираем шрифт
% если Linux Libertine не установлен, то
% можно также попробовать Helvetica, Arial, Cambria и т.Д.

% чтобы использовать шрифт \Linux Libertine на личном компе,
% его надо предварительно скачать по ссылке
% http://www.\Linuxlibertine.org/index.php?id=91&L=1

% на сервисах типа sharelatex.com этот шрифт есть :)

\newfontfamily{\cyrillicfonttt}{Linux Libertine O}
% пояснение зачем нужно шаманство с \newfontfamily
% http://tex.stackexchange.com/questions/91507/

\AddEnumerateCounter{\asbuk}{\russian@alph}{щ} % для списков с русскими буквами
\setlist[enumerate, 2]{label=\asbuk*),ref=\asbuk*} % списки уровня 2 будут буквами а) б) ...

%% эконометрические и вероятностные сокращения
\DeclareMathOperator{\Cov}{Cov}
\DeclareMathOperator{\sCov}{sCov}
\DeclareMathOperator{\sVar}{sVar}
\DeclareMathOperator{\sCorr}{sCorr}
\DeclareMathOperator{\Corr}{Corr}
\DeclareMathOperator{\Var}{Var}
\DeclareMathOperator{\E}{E}
\DeclareMathOperator{\N}{N}
\DeclareMathOperator{\tr}{trace}
\DeclareMathOperator{\trace}{trace}
\DeclareMathOperator{\Lin}{Lin}
\DeclareMathOperator{\Linp}{Lin^{\perp}}
\DeclareMathOperator{\col}{col}
\DeclareMathOperator{\colp}{col^{\perp}}
\DeclareMathOperator{\grad}{grad}

\def \hb{\hat{\beta}}
\def \hs{\hat{\sigma}}
\def \htheta{\hat{\theta}}
\def \s{\sigma}
\def \hy{\hat{y}}
\def \hlambda{\hat{\lambda}}
\def \hbeta{\hat{\beta}}
\def \ha{\hat{a}}
\def \hY{\hat{Y}}
\def \hu{\hat{u}}
\def \v1{\vec{1}}
\def \e{\varepsilon}
\def \he{\hat{\e}}
\def \z{z}
\def \hVar{\widehat{\Var}}
\def \hCorr{\widehat{\Corr}}
\def \hCov{\widehat{\Cov}}
\def \cN{\mathcal{N}}
\def \RR{\mathbb{R}}

\makeatletter
\def\MT@warn@unknown{}
\makeatother

\newcommand{\tvect}[3]{%
   \ensuremath{\Bigl(\negthinspace\begin{smallmatrix}#1\\#2\\#3\end{smallmatrix}\Bigr)}}



\begin{document}

Конспектировал: Антон Захаренков.
\bigskip

\smallskip

\textbf{Вспомним обозначения}


\begin{itemize}[label={$\bullet$}]
        \item $\theta_0$ --- истинное значение параметра
        \item $\theta$ --- произвольное значение параметра
        \item $\htheta$ --- оценка ML
\end{itemize}

Производная логарифмической функции потерь:
$s(\theta) = \grad l(\theta) = \grad \ln L(\theta)$

\begin{equation}
    \Var s(\theta_0)=
    \begin{cases}
      \E ( s(\theta_0) - s(\theta_0)^T ) \\[5pt]
      \E \Big(-\dfrac{\partial^2 l}{\partial \theta \: \partial \theta^T}\Big\rvert_{\theta_0} \Big)
    \end{cases}
  \end{equation}


\smallskip
\textbf{Утверждение.} $\Var s(\theta_0)$ и $\Var (\htheta)$ --- взаимообратны.



\bigskip
\textbf{Пример}

    $y_1, y_2, ..., y_n$ --- выборка независимых случайных величин из распределения Exp($\lambda$)
    
\medskip    
\begin{enumerate}
%\setlength\itemsep{0.1em}
    \item Найдите $\hlambda$ 
    \item Найдите $\Var(\hlambda)$ 
    \item Найдите $\Var(s(\lambda_0))$ 
    \item Найдите $\Var(\hlambda) \cdot \Var(s(\lambda_0))$
\end{enumerate}
    
\medskip    
\emph{Решение}

Экспоненциальное распределение: \begin{equation}
    f(y_0)=
    \begin{cases}
      \lambda e^{-\lambda y_i},\quad y_i \geq 0\\[5pt]
      0,\quad y_i < 0
    \end{cases}
  \end{equation} 
  
Функция правдоподобия: \begin{equation}
    L(y_1, ..., y_n, \theta)=
    \lambda^n e^{-\lambda (y_1 +  ... + y_n)}
  \end{equation}

Логарифмируем: \begin{equation}
    l(y_1, ..., y_n, \theta)=
    n\ln\lambda - \lambda (y_1 +  ... + y_n)
  \end{equation}
  
Дифференцируем:  \begin{equation}
    s(\lambda)=
    \dfrac{n}{\lambda} - (y_1 +  ... + y_n)
  \end{equation}
  
  \begin{equation}
    \hlambda=
    \dfrac{n}{\sum y_i}
  \end{equation}
  
  \begin{equation}
    \Var s(\lambda_0) =
    \E \Big(-\dfrac{\partial^2 l}{\partial \lambda \: \partial \lambda^T}\Big\rvert_{\lambda_0} \Big) = 
    \E \Big(\dfrac{n}{\lambda_0^2}\Big)=\dfrac{n}{\lambda_0^2}
  \end{equation}
  
  Примерно найдем $\Var (\hlambda)$ с помощью дельта-метода.
  
  \medskip
  
 По ЦПТ:\begin{equation}
    \dfrac{\sum y_i}{n} \stackrel{acc}{\sim} \N \Big(\dfrac{1}{\lambda_0};\dfrac{1}{n\lambda_0^2}\Big)
  \end{equation}
  
 Но:\begin{equation}
    \hlambda = \dfrac{1}{\Big(\dfrac{\sum y_i}{n}\Big)} = h\Big(\dfrac{\sum y_i}{n}\Big)
  \end{equation}
   
 Разложим h(t) в ряд Тейлора в окрестности $\dfrac{1}{\lambda}$:
 \begin{equation}
    h(t) = \dfrac{1}{t} = \dfrac{1}{t_0} - \dfrac{1}{t_0^2}(t-t_0)
  \end{equation}
 
 Тогда:
 \begin{equation}
    \hlambda \approx \lambda_0 - \lambda_0^2 \Big(\dfrac{\sum y_i}{n} - \dfrac{1}{\lambda_0} \Big)
\end{equation}

Дисперсия:
\begin{equation}
    \Var (\hlambda) = \lambda_0^4 \cdot \dfrac{1}{n\lambda_0^2} = \dfrac{\lambda_0^2}{n}
  \end{equation}
  
Видим, что:
\begin{equation}
    \Var (\hlambda) \cdot \Var (s(\lambda_0)) \approx \dfrac{\lambda_0^2}{n} \cdot \dfrac{n}{\lambda_0^2} \to 1
  \end{equation}
  
\bigskip
\textbf{Упражнение}

Рассмотрим модель множественной регрессии, $y=X\beta + u$, где регрессоры детерминистические, а $u \sim \cN(0; \sigma^2 \cdot I)$.
Величина $\sigma^2$ известна. 

\begin{enumerate}
\item Найдите $\hbeta_{ML}$.
\item Найдите $\Var (\hbeta_{ML})$.
\item Найдите $s(\beta_0)$.
\item Найдите $\Var (s(\beta_0))$.
\item Найдите $\Var (\hbeta) \cdot \Var (s(\beta_0))$.
\end{enumerate}

\medskip    
\emph{Решение}

По условию:
 \begin{equation}
    y \sim \cN(X\beta; \sigma^2 \cdot I) 
\end{equation}

Функция правдоподобия:
\begin{equation}
    L (y, \beta) = \dfrac{1}{(2\pi \sigma^2)^{\frac{n}{2}}} \cdot exp(-\dfrac{1}{2}(y-X\beta)^{T} (\sigma^2 \cdot I)^{-1} (y-X\beta))
  \end{equation}
  
Логарифмируем:
\begin{equation}
    l (y, \beta) = -\dfrac{n}{2} \: \ln(2\pi \sigma^2) -\dfrac{1}{2}(y-X\beta)^{T} \dfrac{1}{\sigma^2} (y-X\beta) \to \underset{\beta}{max}
  \end{equation}
  
Оценка ML совпадает с оценкой МНК:
\begin{equation}
    \hbeta_{ML} = \hbeta_{MHK} = (X^T X)^{-1} X^{T} y
  \end{equation}
\begin{equation}
    \Var (\hbeta_{ML}) = \sigma^2 (X^T X)^{-1}
  \end{equation}
  
Посчитаем дифференциал dl:
\begin{equation}
   dl = -\dfrac{1}{2\sigma^2} [-(X d\beta)^T (y-X\beta) + (y-X\beta)^T (-X d\beta)] = \dfrac{(y-X\beta)^T X d\beta}{\sigma^2} = s(\beta)^T d\beta
  \end{equation}
  
Отсюда:
\begin{equation}
   s(\beta_0) = \dfrac{1}{\sigma^2} X^T(y-X\beta_0)
  \end{equation} 
\begin{equation}
   \Var (s(\beta_0)) = \dfrac{1}{\sigma^4} X^T \Var(y-X\beta_0) X = \dfrac{X^T X}{\sigma^2}
  \end{equation} 
  
Видим, что:
\begin{equation}
   \Var (\hbeta) \cdot \Var (s(\beta_0) = 1
  \end{equation} 


\bigskip
\textbf{Замечание}

При [некоторые условия регулярности]:
\begin{enumerate}
    \item plim $\htheta_{ML} = \theta_0$
    \item $\htheta_{ML} \sim \cN(\theta;  \Var(s(\theta_0))^{-1})$
    \item $\sqrt{n} \: (\htheta_{ML} - \theta_0) \stackrel{d}{\to} \cN(0; n \cdot \Var(s(\theta_0))^{-1})$
\end{enumerate}

Откуда:

\begin{center}
\fbox{\begin{varwidth}{\dimexpr\textwidth-10\fboxsep-10\fboxrule\relax}
Если $\ha \sim \cN(a; \Var(\ha)$, то \qquad

\[
 (\ha - a)^T \Var(\ha)^{-1} (\ha - a) \sim \chi_k^2
\]
\end{varwidth}}
\end{center}

\bigskip
\textbf{Проверка гипотезы о равенстве параметров заданным значениям}

\begin{itemize}
    \item $H_0: \theta = \theta_0$
    \item $H_A: $ хотя бы одно из равенств нарушено
\end{itemize}

Если $H_0$:
\begin{equation}
  W = (\htheta_{UR} - \theta_0)^T \Var (\htheta_{UR})^{-1} (\htheta_{UR} - \theta_0) \sim \chi_k^2
  \end{equation} 
\begin{equation}
   LM = (\htheta_{R})^T \Var (s(\htheta_{R}))^{-1}  (\htheta_{R}) \sim \chi_k^2
  \end{equation} 
\begin{equation}
   LR = 2\cdot (l(\htheta_{UR}) - l(\htheta_{R})) \sim \chi_k^2
  \end{equation} 
  
\newpage
Как состоятельно оценить $\Var (s(\theta_0) ? =\E \Big(-\dfrac{\partial^2 l}{\partial \theta \: \partial \theta^T}\Big\rvert_{\theta_0} \Big)$
\begin{enumerate}
    \item Подставить $\htheta$ вместо $\theta_0$
    \item Можно даже не брать E
\end{enumerate}

\bigskip
\textbf{Пример}

    $y_1, ..., y_n$ --- выборка независимых случайных величин из распределения Exp($\lambda$), n=100, $\sum y_i = 400$
\medskip    
\begin{itemize}
    \item $H_0: \lambda = 0.3$
    \item $H_A: \lambda \neq 0.3$
\end{itemize}

Проверить с помощью:
\medskip    
\begin{enumerate}

    \item Wald
    \item LM
    \item LR
\end{enumerate}

\medskip    
\emph{Решение}

\begin{equation}
   \hlambda_{ML} = \dfrac{n}{\sum y_i} = \dfrac{100}{400} = 0.25
  \end{equation} 
\begin{equation}
   \Var (\hlambda_{ML})^{-1} \approx \Var (s(\lambda_0)) = \dfrac{4}{\lambda_0^2} = \dfrac{100}{0.25^2} = 1600
  \end{equation} 
  
\begin{equation}
   s(\lambda_0) = \dfrac{n}{\lambda_0} - \sum y_i = {100}{0.3} - 400 = -\dfrac{200}{3}
  \end{equation} 

\begin{equation}
   \Var s(\lambda_0) = \dfrac{n}{\lambda_0^2} = \dfrac{100}{0.3^2}
  \end{equation}   
  

\begin{equation}
   W = (0.25-0.3)\cdot 1600 \cdot (0.25-03) = 4 < \chi_1^2 = 6.63 \Rightarrow H_0 - not rejected
  \end{equation}
  
\begin{equation}
   LM = (-\dfrac{200}{3}) \cdot \dfrac{1}{\frac{100}{0.3^2}} \cdot (-\dfrac{200}{3}) = 4 < \chi_1^2 = 6.63 \Rightarrow H_0 - not rejected
  \end{equation}
  
\newpage

\bigskip
\textbf{Упражнение}

В пруду водятся

\begin{table}[htb]
\centering
\begin{tabular}{l|l|l|}
\cline{2-3}
                              & \begin{tabular}[c]{@{}l@{}}Вероятность \\ обнаружения\end{tabular} & Было выловлено \\ \hline
\multicolumn{1}{|l|}{щуки}    & p                                                                  & 10             \\ \hline
\multicolumn{1}{|l|}{русалки} & r                                                                  & 30             \\ \hline
\multicolumn{1}{|l|}{водяные} & v                                                                  & 20             \\ \hline
\multicolumn{1}{|l|}{караси}  & 1-p-r-v                                                            & 140            \\ \hline
\end{tabular}
\end{table}
   
причем каждое выловленное животное не влияет на распределение последующих.

Проверяется гипотеза:
\medskip    
\begin{itemize}
    \item $H_0: $  \begin{cases}
      r = v \\[2pt]
      p = 0.1
    \end{cases}
    \item $H_A: $ хотя бы что-то невыполнено
\end{itemize}

\medskip    
\emph{Решение}

Пусть
\begin{equation}
    H_0 =  \begin{cases}
      r - v = \alpha = 0 \\[2pt]
      p - 0.1 = \beta = 0
    \end{cases}
\end{equation}

Правдоподобие:
\begin{equation}
    L = const \cdot (0.1 + \beta)^{10} \cdot (v+\alpha)^{30} \cdot v^{20} \cdot (0.9 - \beta - 2v - \alpha)^{140}
\end{equation}

\begin{equation}
    l = const + 10 \ln (0.1 + \beta) + 30 \ln (v+\alpha) + 20 \ln v +  140 \ln (0.9 - \beta - 2v - \alpha)
\end{equation}

\bigskip
Пусть $\theta = \tvect{\alpha}{\beta}{v}$: 

\begin{align}
    s (\theta) &=  \begin{pmatrix}
           30\dfrac{1}{v+\alpha} - 140\dfrac{1}{0.9-\beta-2v-\alpha} \\
           \hdots \\
           \hdots \\
         \end{pmatrix}
  \end{align}

\begin{align}
    \hat{\begin{pmatrix}
           p \\
           r \\
           v \\
         \end{pmatrix}}_{UR} = \begin{pmatrix}
           0.1 + \hbeta \\
           \hat{v} + \hlambda \\
           \hat{v} \\
         \end{pmatrix} = \begin{pmatrix}
           \frac{10}{200} \\
           \frac{30}{100} \\
           \frac{20}{200} \\
         \end{pmatrix} = \begin{pmatrix}
           0.05 \\
           -0.05 \\
           0.1 \\
         \end{pmatrix}
  \end{align}


\begin{align}
    \hat{\begin{pmatrix}
           p \\
           r \\
           v \\
         \end{pmatrix}}_{R} = \begin{pmatrix}
           0.1  \\
           \hat{v} \\
           \hat{v} \\
         \end{pmatrix} =\begin{pmatrix}
           0.1 \\
           \frac{45}{380} \\
           \frac{45}{380} \\
         \end{pmatrix}
  \end{align}




\end{document}
